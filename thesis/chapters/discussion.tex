This chapter discusses the relevance of \bpfbox{} and \bpfcontain{}, positioning them as
novel extensions on top of the existing confinement literature. We also examine
limitations of both research systems and present opportunities for future work. Namely, we
propose ways to address current limitations, improve the \gls{ebpf} ecosystem for
confinement use cases, add features to \bpfbox{} and \bpfcontain{}, and conduct further
research on the usability of both systems.

\section{Improving the Status Quo}%
\label{s:disc-improving}

\todo{This section will discuss how \bpfbox{} and \bpfcontain{} improve upon the status
quo in confinement. Specifically, it will compare both systems with related work.}

\begin{inprogress}
  Application-specific and container-specific policies
  \begin{itemize}
    \item \bpfbox{} policies are \textit{application-specific}
    \item Specifically, \bpfbox{} doesn't try to enforce policy over the entire system like AppArmor or SELinux
    \item Instead, we focus on ad-hoc confinement of a specific application
    \item This affords the user a great deal of flexibility; no need to author complex policies that cover the entire system. Instead, focus on the specific behaviours you want your application to exhibit
    \item In turn, this flexibility means that different types of users can leverage \bpfbox{} in different ways. Application developers can write fine-grained policies that enforce behaviours at the function-call-level. End-users can deploy custom \bpfbox{} policies to restrict specific behaviours, such as access to the home directory.

    \item \bpfcontain{} extends the \bpfbox{} model to be \textit{container-specific}
    \item Like \bpfbox{}, a lot of the implicit strength behind this approach is that policies are not trying to enforce system-wide behaviour
    \item But the container-specific approach also affords even more implicit advantages
    \item First, it is designed to be applicable to the container use-case, unlike traditional \gls{lsm} which are designed to enforce generic confinement policy on the whole system
    \item Using container semantics, we can avoid the need to use overly coarse-grained policy templates, instead focusing on protecting the container at a per-resource level
    \item We can significantly simplify policy and improve container security by defining a clear protection boundary around the container
    \item User-defined policies then focus on defining specific exceptions to the security boundary
    \item This mirrors the idea of provisioning shared resources in a virtual machine
  \end{itemize}

  \gls{ebpf} and adoptability
  \begin{itemize}
    \item Adoptability that comes with \gls{ebpf} compared with traditional \glspl{lsm}
    \item Companies are already using \gls{ebpf} in production for performance monitoring use cases
    \item The KRSI patch that provides \gls{lsm} probes was invented at Google for custom policies and security auditing in their production servers
    \item The barrier to entry for running new \gls{ebpf} code on a production system is much lower than a new kernel module or kernel patch
    \item \gls{ebpf} kernel code is far less likely to have adverse side effects, security vulnerabilities

    \item Another advantage of an \gls{ebpf}-based solution over traditional solutions is that it is far easier to extend and modify
    \item Rapidly prototype, test, and deploy new kernel security mechanisms
    \item Easy to modify \bpfbox{} and \bpfcontain{} to support new rule types, employ different policy defaults, and to include other aspects of system state
    \item In the future, we will likely see \gls{ebpf} being increasingly used to develop new and innovative security solutions, combining
    system observability with policy enforcement
  \end{itemize}

  Simple yet expressive policies encourage local policy variation
  \begin{itemize}
    \item Diversity in computer security is an area that is now seeing increasing exploration (cite some of Anil's papers)
    \item Increasing the diversity of software deployments can make it difficult for an attacker to re-use knowledge of one
    system configuration to attack another. For example, software configuration might have a particular vulnerability, but
    another might have a totally different vulnerability. By invalidating attacker knowledge and assumptions, we can preemptively
    make it much harder to execute successful attacks at scale

    \item Classically, researchers have investigated source-level or binary-level variations as a source of diversity
    \item However, these are often quite baroque and have limited practical use, since diversity at the source or binary level can
    lead to difficulties in scalable deployment, reliability, and the effectiveness of software
    \item This notion goes against much of what we want to do as computer scientists: have a system be as predictable and reproducible as possible

    \item However, security policies make good sense as a layer to introduce diversity. By diversifying security policies, we can
    make the exploitation targets less reliable and consistent for attackers, without necessarily impacting the usefulness of an application.
    For instance, one end-user of a piece of software might only need a certain subset of its functionality. In this case, their security policy
    might look very different from another end-user that requires a different subset of functionality. Introducing variations in security policy
    for this software can confound attackers, much in the same way that source-level or binary-level variations would.
    \item \bpfbox{} and \bpfcontain{} are ideal security mechanisms for encouraging this kind of policy variation,
    because they are designed for a local administrator to tailor a security policy to precisely the set of desired functionality
  \end{itemize}
\end{inprogress}


\section{Limitations}%
\label{s:disc-limitations}

In this section, we discuss some limitations of the \bpfbox{} and \bpcontain{}.
While some limitations arise due to a lack of support for the correct primitives
in the current \gls{ebpf} ecosystem, others arise due to the prototypical nature of \bpfbox{}
and \bpfcontain{} as research systems. In both cases, we discuss how future iterations
of \bpfbox{} and \bpfcontain can address these limitations, either through extensions to
the policy enforcement mechanism or future improvements to the \gls{ebpf} ecosystem.

\subsection{Pathname Support in \glsentryshort{ebpf}}

\begin{inprogress}
  \begin{itemize}
    \item Hard to refer to files from \gls{ebpf}
    \item Currently, \bpfbox{} and \bpfcontain{} translate file policy into inodes and filesystem device IDs
    \item This is a crude workaround; it has some convenient side effects for security,
    but issues arise when we want to refer to pathnames that don't exist yet
    \item It can also cause an explosion in the size of maps storing file rules, as
    globbed paths get translated into multiple rules: one for each file that matches the
    glob

    \item The difficulty working with pathnames is partially a result of a fundamental limitation of \gls{ebpf}: difficulty manipulating strings
    \item Problem generally arises from three factors, primarily related to the verifier:
    \begin{itemize}
      \item Verifier imposes a hard 512 byte limit on stack allocations (strings need to be heap-allocated, stored in an map)
      \item Verifier imposes restrictions on how programs can loop (looping needs to provably terminate, the verifier errs on the side of caution here)
      \item Helper functions can get around these restrictions, but decent string and pathname helpers are no here yet
    \end{itemize}
    \item Another fundamental issue is that support for sleepable \gls{bpf} is new and has
    not yet matured (only a small subset of \gls{lsm} programs can currently be marked
    sleepable)
    \item Linux 5.(11?) added \texttt{bpf\_d\_path}, but this is only callable from sleepable programs, a subset of \gls{lsm} programs
    \item In the current \bpfbox{} and \bpfcontain{} design, this limitation is too restrictive
    \item Luckily, it seems like the community is working towards a general solution to
    this problem (dynamic map allocation and making more program types sleepable)
    \item As the \gls{ebpf} ecosystem evolves, it may be possible to support pathnames as
    a first class citizen, removing the requirement for working with inodes and filesystem
    numbers
  \end{itemize}
\end{inprogress}

\subsection{Fixed-Size Policy Maps}

\begin{inprogress}
  \begin{itemize}
    \item Currently, policy maps are of a fixed size
    \item It's okay to make them big, since \gls{ebpf} does support map growth up to a fixed limit
    \item But we are still limited in total map size (todo: get current figure)
    \item In our current implementation, we simply grow policy maps from userspace when the map size would be too small to fit current rules
    \item But this approach is still limited, as it doesn't support map resizing at runtime, only at load time
    \item Once sleepable \gls{bpf} matures, we can have dynamically allocated maps of
    arbitrary size at runtime (link Alexei's LKML thread), as we can now have runtime map
    allocators where is it okay to sleep on a page fault
  \end{itemize}
\end{inprogress}

\subsection{Passive Performance Overhead}

\begin{inprogress}
  \begin{itemize}
    \item Currently, \bpfbox{} and \bpfcontain{} are competitive with mainstream
    confinement solutions based on \gls{lsm} (e.g.~AppArmor, see chapter 6)
    \item This competitiveness is actually an advantage, considering that \bpfbox{} and
    \bpfcontain{} can be dynamically loaded and attached to various system events. In this sense,
    we are getting increased flexibility without paying much of a cost in performance.
    \item But, \bpfbox{} and \bpfcontain{} have the potential to be far more performant than
    conventional \gls{lsm}s in one critical case: passive overhead on the rest of the system.
    \item Due to a current limitation of how KRSI works, its \gls{ebpf} \gls{lsm} hooks are
    always globally invoked, regardless of whether the target process is of interest to us or not.
    \item The current pattern looks like (Invoke Syscall $\rightarrow$ Invoke Hook
    $\rightarrow$ Invoke BPF Program $\rightarrow$ Filter Logic $\rightarrow$ Return from
    BPF Program $\rightarrow$ Return from Hook $\rightarrow$ Return from Syscall).
    \item In the future, we may be able to move the filter logic to the step
    \textit{before}  the hook is called, or at the very least before the BPF program is
    called. This would nearly eliminate any passive overhead on the unconfined parts of the system.
    \item I have a plan for this: introducing a new namespace for \gls{bpf} programs and
    maps. (Forward ref to BPF namespace and unprivileged BPF)
  \end{itemize}
\end{inprogress}

\subsection{Network Policy Granularity}

\begin{inprogress}
  \begin{itemize}
    \item Network policy in \bpfbox{} and \bpfcontain{} is currently very coarse-grained
    \item Only operates at the socket level, and does not considered nuanced access
    controls such as at the per-IP-address level.
    \item This means that specifying access to the network essentially gives the process or container
    access to the global network.
    \item While this is not problematic for applications that do not require network access,
    it quickly becomes an overprivilege issue for applications that do.
    \item To fix this, we can introduce a finer-grained network policy that specifies per-IP network access, an extension which
    is possible using currently available \gls{ebpf} technology. (Forward ref to protocol-level network policy)
  \end{itemize}
\end{inprogress}


\section{Future Work and Research Directions}%
\label{s:disc-future-work}

\todo{This section will discuss opportunities for future work.}

\subsection{Policy Language Experimentation and Usability Study}

\begin{inprogress}
  Policy language experimentation
  \begin{itemize}
    \item There is room for further policy language experimentation
    \item Perhaps \bpfbox{} would benefit from a similar policy language schema to \bpfcontain{}, rather than a DSL
    \item Or perhaps \bpfcontain{} would benefit from some alternate policy languages (different schemas, different serialization formats)
    \item Experiment with policy granularity, tunables that impact \bpfcontain{}'s default enforcement boundary
  \end{itemize}

  User study
  \begin{itemize}
    \item Determine quantitatively how \bpfcontain{}'s
    \item Introduce three research questions
    \begin{itemize}
      \item RQ1: How well do \bpfcontain{}'s policy semantics match user expectations?
      \begin{itemize}
        \item Does \bpfcontain{}'s default policy fit the user's mental model of a container? Any surprises?
        \item Is the \bpfcontain{} policy language suitable for the kind of confinement users want to do?
      \end{itemize}
      \item RQ2: How does the user experience of \bpfcontain{} compare with alternatives?
      \begin{itemize}
        \item Compare with alternative \glspl{lsm} (SELinux and AppArmor), seccomp-bpf, perhaps higher-level mechanisms like Snap
      \end{itemize}
      \item RQ3: How do users perceive the security of \bpfcontain{} compared with alternatives?
      \begin{itemize}
        \item Compare with alternative \glspl{lsm} (SELinux and AppArmor), seccomp-bpf, perhaps higher-level mechanisms like Snap
      \end{itemize}
    \end{itemize}
    \item Cite the FBAC-LSM paper for some ideas on how to conduct the user study
  \end{itemize}

  User study can inform further policy language experimentation
  \begin{itemize}
    \item Use answers to research questions to inform further design
    \item Adding new rule types, modifying existing rule types
    \item Change the design of the policy language, add more tunables, etc.
  \end{itemize}
\end{inprogress}

\subsection{The BPF Namespace and Unprivileged BPF}

\begin{inprogress}
  \begin{itemize}
    \item
  \end{itemize}
\end{inprogress}

\subsection{OCI and Docker Integration}

\begin{inprogress}
  \begin{itemize}
    \item
  \end{itemize}
\end{inprogress}

\subsection{Fine-Grained Network Policy}

\begin{inprogress}
  The problem with existing network policy
  \begin{itemize}
    \item \bpfbox{} and \bpfcontain{} currently support very coarse-grained networking policy
    \item They enforce network access control at the socket level; a process of container
    either has access to specific socket operations or they don't
    \item But this approach doesn't account for many of the nuances of network policy
    \item Allowing a process to use networking capability essentially opens it up to the outside world
    \item Further, it is often desirable to only have a container's network interface exposed to the local system
    \item Since policies lock down access to the rest of the system, this isn't as severe of a problem as it could be... Remote adversaries are still confined to accessing specific system resources according to the confinement policy
    \item But it would still be better to support much finer-grained networking policy from a least-privilege point of view
    \item Eliminate the need for a netfilter firewall (e.g.~iptables) for securing container's networking stack
  \end{itemize}

  How network policy could be extended
  \begin{itemize}
    \item \gls{ebpf} \texttt{TC\_CLSACT} program
    \item Stands for \enquote{traffic classifier and actions}
    \item Can be attached to a network interface, handles both ingress and egress
    \item Works similar in spirit to classic \gls{bpf} packet filters, but additionally supports making policy decisions and modifying packets
    \item We can use this program type, attached to a container's networking interface, to match specific characteristics of network traffic,
    such as IP address patterns, port numbers, and other protocol-level details
    \item The security policy itself cam be extended to support these characteristics in addition to basic socket operations
    \item We can then match this information against the container's security policy, and achieve much finer-grained network policy enforcement
  \end{itemize}
\end{inprogress}

\subsection{\bpfcontain{} Policy Generation and \glsentryshort{gui}}

\begin{inprogress}
  \begin{itemize}
    \item
  \end{itemize}
\end{inprogress}



\section{Conclusion}
\label{s:disc-conclusion}

\todo{This section will conclude the thesis, highlighting the important aspects of
\bpfbox{} and \bpfcontain{} and contributions}
