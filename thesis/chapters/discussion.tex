This chapter discusses the relevance of \bpfbox{} and \bpfcontain{}, positioning them as
novel extensions on top of the existing confinement literature. We also examine
limitations of both research systems and present opportunities for future work. Namely, we
propose ways to address current limitations, improve the \gls{ebpf} ecosystem for
confinement use cases, add features to \bpfbox{} and \bpfcontain{}, and conduct further
research on the usability of both systems.

\section{Improving the Status Quo}%
\label{s:disc-improving}

In this section, we discuss how \bpfbox{} and \bpfcontain{} improve upon the status quo in
confinement, with a particular emphasis on how their unique properties encourage
application- and container-specific confinement, and promote local policy variation. We
also examine how their implementation as \gls{ebpf}-based security solutions positions
them as highly adoptable alternatives to traditional kernel security mechanisms, with the
potential to drive further innovation going forward.

\subsection{Application-Specific and Container-Specific Policies}

Due to their simplicity and flexibility, \bpfbox{} and \bpfcontain{} encourage a different
kind of confinement compared with existing \gls{lsm}-based solutions. Rather than focusing
on global, system-wide \gls{mac} policy enforcement like
SELinux~\cite{smalley2001_selinux} or AppArmor~\cite{cowan2000_apparmor}, \bpfbox{} and
\bpfcontain{} focus on application-specific and container-specific confinement
respectively. The former enables light-weight, ad-hoc confinement of individual Linux
processes, while the latter extends this model to work with container semantics.  This
represents a stark contrast over existing work in the confinement space, which generally
focuses on reusing existing primitives designed for global enforcement.  Rather than
outright simple policies, these confinement frameworks compile down into hundreds or even
thousands of lines of policy for the underlying confinement primitive.

The \textit{application-specific} approach of \bpfbox{} takes a different path\,---\,a
\bpfbox{} policy directly applies to the \bpfbox{} enforcement engine, without relying on
existing primitives like SELinux, AppArmor, or Seccomp-bpf. The result is policies that
are far simpler without being too coarse-grained or abstracted beyond the point of
auditability\,---\,expressiveness without complexity. Further, application-specific
policies afford a great deal of flexibility to the end-user; rather than authoring complex
policies that cover the entire system, instead they may focus on the specific behaviours
that they want their applications to exhibit. In turn, this flexibility means that
different types of users can leverage \bpfbox{} in different ways.  Application developers
can write fine-grained policies that enforce behaviours at the function-call-level.
Conversely, end-users can deploy custom \bpfbox{} policies to restrict specific
behaviours, such as access to the home directory.

\bpfcontain{} extends the \bpfbox{} model to be \textit{container-specific}. Like
\bpfbox{}, much of the implicit strength inherent in this model is that it does not rely
on existing primitives and does not attempt to enforce policy over the entire system.
Instead, we focus on individual containers, identifying the specific \gls{os} interfaces
needed for the container to function.  This approach introduces additional advantages
specific to the container use case, on top of the advantages already present in the
\bpfbox{} design. By including container semantics as part of its internal model of the
system, \bpfcontain{} can enforce highly nuanced policy defaults, defining an enforcement
boundary around the container. This boundary, in turn, simplifies resulting policies by
allowing the policy author to focus on which external interfaces the container needs,
without worrying about specifying access to internal resources. In this way, \bpfcontain{}
policies mirror the process of provisioning resources in a virtual machine.

% \begin{inprogress}
%   Application-specific and container-specific policies
%   \begin{itemize}
%     \item \bpfbox{} policies are \textit{application-specific}
%     \item Specifically, \bpfbox{} doesn't try to enforce policy over the entire system like AppArmor or SELinux
%     \item Instead, we focus on ad-hoc confinement of a specific application
%     \item This affords the user a great deal of flexibility; no need to author complex policies that cover the entire system. Instead, focus on the specific behaviours you want your application to exhibit
%     \item In turn, this flexibility means that different types of users can leverage \bpfbox{} in different ways. Application developers can write fine-grained policies that enforce behaviours at the function-call-level. End-users can deploy custom \bpfbox{} policies to restrict specific behaviours, such as access to the home directory.

%     \item \bpfcontain{} extends the \bpfbox{} model to be \textit{container-specific}
%     \item Like \bpfbox{}, a lot of the implicit strength behind this approach is that policies are not trying to enforce system-wide behaviour
%     \item But the container-specific approach also affords even more implicit advantages
%     \item First, it is designed to be applicable to the container use-case, unlike traditional \gls{lsm} which are designed to enforce generic confinement policy on the whole system
%     \item Using container semantics, we can avoid the need to use overly coarse-grained policy templates, instead focusing on protecting the container at a per-resource level
%     \item We can significantly simplify policy and improve container security by defining a clear protection boundary around the container
%     \item User-defined policies then focus on defining specific exceptions to the security boundary
%     \item This mirrors the idea of provisioning shared resources in a virtual machine
%   \end{itemize}
% \end{inprogress}

\subsection{Encouraging Local Policy Variation}

An implicit advantage of simple yet flexible confinement policies is that they encourage
\textit{policy variation}. In computer security, \textit{diversity} is an area that has
seen some exploration in the past~\cite{somayaji2007_immunology, persaud2016_frankenssl,
matrawy2005_mitigating, neti2012_software}. The idea is deceptively simple, inspired by
biological sources of diversity found in nature. Attackers rely on similarities between
system for widespread exploitation; for instance, consider a set of deployed systems all
running the same vulnerable version of a piece of a software with similar configurations.
Since each of these systems have the same vulnerability, exploiting one system is much
like exploiting the others; thus, the knowledge and effort required to exploit each system
will be roughly equivalent to exploiting just one. The point of introducing diversity into
software deployments and configurations is to confound this basic assumption. On
a macroscopic scale, increased diversity improves the security of the entire population.

Classically, computer diversity has primarily been explored through source- or
binary-level variations~\cite{somayaji2007_immunology, persaud2016_frankenssl,
neti2012_software}. Solutions recombine existing software in new ways, such that each
deployment has a unique code footprint. However, a primary limiting factor to the adoption
of such security solutions is that they fly in the face of the traditional approach to
computing. Namely, we want our software to be as predictable, reliable, and effective as
possible. The same input into a piece of software should produce the same (or at least
predictable) output, regardless of the underlying system configuration. Source- or
binary-level software diversity invalidates this assumption, by introducing potential
variations in common code paths.

Unlike traditional sources of diversity in computing, \textit{policy-level diversity} has
the potential to be quite effective while maintaining the underlying assumption that
software should \enquote{just work}. Security policies may be tailored to some
locally-desired use case, only invalidating code paths that would never be taken to begin
with. For example, consider an Apache web server configuration. In one deployment, it may
be necessary to support the execution of \gls{cgi} helper scripts, whereas another may
only need to serve static webpages. Each configuration could use a different security
policy, based on the needs of the local deployment. A natural diversity arises as an
emergent property of this model when scaled up over thousands of deployments across
thousands of applications.  Since \bpfbox{} and \bpfcontain{} policies are so simple yet
so flexible, they naturally encourage precisely the soft of local policy variations
required to achieve policy-level diversity at scale.


% \begin{inprogress}
%   Simple yet expressive policies encourage local policy variation
%   \begin{itemize}
%     \item Diversity in computer security is an area that is now seeing increasing exploration (cite some of Anil's papers)
%     \item Increasing the diversity of software deployments can make it difficult for an attacker to re-use knowledge of one
%     system configuration to attack another. For example, software configuration might have a particular vulnerability, but
%     another might have a totally different vulnerability. By invalidating attacker knowledge and assumptions, we can preemptively
%     make it much harder to execute successful attacks at scale

%     \item Classically, researchers have investigated source-level or binary-level variations as a source of diversity
%     \item However, these are often quite baroque and have limited practical use, since diversity at the source or binary level can
%     lead to difficulties in scalable deployment, reliability, and the effectiveness of software
%     \item This notion goes against much of what we want to do as computer scientists: have a system be as predictable and reproducible as possible

%     \item However, security policies make good sense as a layer to introduce diversity. By diversifying security policies, we can
%     make the exploitation targets less reliable and consistent for attackers, without necessarily impacting the usefulness of an application.
%     For instance, one end-user of a piece of software might only need a certain subset of its functionality. In this case, their security policy
%     might look very different from another end-user that requires a different subset of functionality. Introducing variations in security policy
%     for this software can confound attackers, much in the same way that source-level or binary-level variations would.
%     \item \bpfbox{} and \bpfcontain{} are ideal security mechanisms for encouraging this kind of policy variation,
%     because they are designed for a local administrator to tailor a security policy to precisely the set of desired functionality
%   \end{itemize}
% \end{inprogress}

\subsection{\glsentryshort{ebpf}, Adoptability, and Future Innovation}

A third improvement over the status quo in confinement lies in \bpfbox{} and
\bpfcontain{}'s novelty as \gls{ebpf}-based confinement implementations. Whereas existing
kernel security mechanisms are based on kernel patches or loadable kernel modules,
\bpfbox{} and \bpfcontain{} leverage \gls{ebpf} for dynamic runtime security monitoring,
verified for safety and correctness before being loaded into the kernel. An
\gls{ebpf}-based implementation affords \bpfbox{} and \bpfcontain{} a number of advantages
over traditional security solutions.

\begin{enumerate}
  \item \textbf{\gls{ebpf} is already widely-used in production Linux environments.} At
  the time of writing this thesis, major software companies like Facebook, Netflix, and
  Google~\cite{gregg2019_bpf} already use \gls{ebpf} in production servers for performance
  and security monitoring, the implementation of L4 routing and load balancing algorithms,
  and other various use cases. In fact, the KRSI patch that provides the \gls{lsm} probes
  used by \bpfbox{} and \bpfcontain{} was initially developed at Google for dynamic
  security monitoring use cases~\cite{singh2019_krsi}. As time goes on, \gls{ebpf}-based
  kernel code is seeing increasing deployment across multidisciplinary areas of industry.
  This widespread deployment is a key incentive toward the adoption of security mechanisms
  like \bpfbox{} and \bpfcontain{}. Not only is this widespread adoption an emergent
  phenomenon from \gls{ebpf}'s safety and flexibility properties, but it is also
  a valuable case study in how \gls{ebpf} is reshaping the way we think about kernel
  observability in practice.

  \item \textbf{The barrier to entry for running new \gls{ebpf} code on a production
  system is much lower than a new kernel module or kernel patch.}
  Since \gls{ebpf} programs and maps can be dynamically loaded into the kernel and are
  checked for safety and correctness before they are allowed to run, the barrier to entry
  for running a new \gls{ebpf}-based implementation in production is far lower than that
  of a kernel module or kernel patch. \gls{ebpf} programs are far less likely to contain
  memory safety errors or other software bugs that plague kernel code in practice. Due to
  new technologies like \gls{bpf} \gls{core}~\cite{nakryiko2020_core}, \gls{ebpf}-based
  solutions are also far more portable across different kernel versions and
  configurations. These factors combined make \bpfbox{} and particularly \bpfcontain{} far
  more adoptable as novel kernel security mechanisms. Future security implementations
  based on \gls{ebpf} can also enjoy these advantages.

  \item \textbf{\gls{ebpf} enables rapid prototyping and deployment of kernel security mechanisms.}
  Due to its dynamic nature, safety, and portability across kernels, it is far easier to
  rapidly prototype, test, and deploy novel security solutions based on \gls{ebpf}. In the
  case of \bpfbox{} and \bpfcontain{}, this enables rapid prototyping of the policy
  language and enforcement engine, and makes it easy to incorporate novel extensions on
  top of these research systems, redefining key aspects of policy enforcement and adding
  new data sources from the kernel and userspace programs. In the future, we will likely see
  \gls{ebpf} positioned as a key enabling factor behind the rapid development of novel kernel
  security extensions, further combining system observability with dynamic policy enforcement.
\end{enumerate}

% \begin{inprogress}
%   \gls{ebpf} and adoptability
%   \begin{itemize}
%     \item Adoptability that comes with \gls{ebpf} compared with traditional \glspl{lsm}
%     \item Companies are already using \gls{ebpf} in production for performance monitoring use cases
%     \item The KRSI patch that provides \gls{lsm} probes was invented at Google for custom policies and security auditing in their production servers
%     \item The barrier to entry for running new \gls{ebpf} code on a production system is much lower than a new kernel module or kernel patch
%     \item \gls{ebpf} kernel code is far less likely to have adverse side effects, security vulnerabilities

%     \item Another advantage of an \gls{ebpf}-based solution over traditional solutions is that it is far easier to extend and modify
%     \item Rapidly prototype, test, and deploy new kernel security mechanisms
%     \item Easy to modify \bpfbox{} and \bpfcontain{} to support new rule types, employ different policy defaults, and to include other aspects of system state
%     \item In the future, we will likely see \gls{ebpf} being increasingly used to develop new and innovative security solutions, combining
%     system observability with policy enforcement
%   \end{itemize}
% \end{inprogress}


\section{Limitations}%
\label{s:disc-limitations}

In this section, we discuss some limitations of the \bpfbox{} and \bpfcontain{}.  While
some limitations arise due to a lack of support for the correct primitives in the current
\gls{ebpf} ecosystem, others arise due to the prototypical nature of \bpfbox{} and
\bpfcontain{} as research systems. In both cases, we discuss how future iterations of
\bpfbox{} and \bpfcontain{} can address these limitations, either through extensions to
the policy enforcement mechanism or future improvements to the \gls{ebpf} ecosystem.

\subsection{Limited Pathname Support in \glsentryshort{ebpf}}

It is hard to refer to files from \gls{ebpf}. In the kernel, files are generally uniquely
described by an \textit{inode} structure, which in turn maps to one or more pathnames via
a \textit{file} structure. Each inode belongs to a distinct filesystem, and is uniquely
enumerated within that filesystem by an inode number.  In \bpfbox{} and \bpfcontain{}, we
uniquely identify inodes using a combination of their inode number and the unique device
identifier of the filesystem on which the inode resides. While this is an effective
technique for runtime monitoring, things begin to fall apart when dealing with
a \textit{user-facing} data store, such as a policy map.

While the kernel refers to files by their inodes within a filesystem, users do not. For
the most part, userspace does not deal in inode-level semantics\,---\,instead, we deal in
\textit{pathnames}, a string that describes the path required to move from the filesystem
root to a given file. Indeed, the \bpfbox{} and \bpfcontain{} policy languages use
pathnames rather than inodes to refer to files. Unfortunately, this creates an undesired
dichotomy between the user-facing components of \bpfbox{} and \bpfcontain{}, and the
kernelspace implementation.  To resolve this dichotomy, we translate the pathnames into
inode and device pairs at policy load-time. This is a workaround, subject to several
fundamental limitations. In particular, referring to a pathname that doesn't yet exist
becomes difficult, as inode numbers do not yet exist; inodes that are deleted or freshly
created at runtime must be treated as special cases, dynamically updating the policy as
required; finally, globbing pathnames can result in an explosion in the size of maps
storing file rules, as each globbed file is translated into a unique inode-device pair.

To resolve these issues, it would be ideal if we could refer to pathnames directly from
\gls{bpf} programs. In particular, a design using this capability might resolve pathnames
within \gls{ebpf} programs and define a finite state machine to match globbing rules over
the pathname. Unfortunately, current support for pathname resolution in \gls{ebpf} is primitive.
Difficulties arise due to a few fundamental limitations imposed by the verifier and the \gls{ebpf}
runtime:
\begin{enumerate}
  \item The verifier imposes a hard limit of 512 bytes of stack space for each \gls{bpf}
  program. This makes it unrealistic to store strings on the stack, instead requiring that
  a buffer be allocated in the heap. In the context of \gls{bpf}, this can only be done
  using a dummy map as a scratch buffer.

  \item The verifier also imposes restrictions on how \gls{ebpf} programs can loop and how
  these loops can access map data. Specifically, loops must provably terminate and any
  array access within a loop must be appropriately bounded by a fixed constant (to ensure
  no buffer overflows or similar issues). In practice, enforcing these restrictions is
  difficult, and the verifier errs on the side caution when reasoning about a loop is unclear.
  This can result in safe programs that manipulate long strings being erroneously rejected.

  \item While helper functions can get around such restrictions, the current ecosystem for
  string manipulation helpers in \gls{ebpf} is immature. For instance, Linux \todo{Continue}
\end{enumerate}


\begin{inprogress}
  \begin{itemize}
    \item Hard to refer to files from \gls{ebpf}
    \item Currently, \bpfbox{} and \bpfcontain{} translate file policy into inodes and filesystem device IDs
    \item This is a crude workaround; it has some convenient side effects for security,
    but issues arise when we want to refer to pathnames that don't exist yet
    \item It can also cause an explosion in the size of maps storing file rules, as
    globbed paths get translated into multiple rules: one for each file that matches the
    glob

    \item The difficulty working with pathnames is partially a result of a fundamental limitation of \gls{ebpf}: difficulty manipulating strings
    \item Problem generally arises from three factors, primarily related to the verifier:
    \begin{itemize}
      \item Verifier imposes a hard 512 byte limit on stack allocations (strings need to be heap-allocated, stored in an map)
      \item Verifier imposes restrictions on how programs can loop (looping needs to provably terminate, the verifier errs on the side of caution here)
      \item Helper functions can get around these restrictions, but decent string and pathname helpers are no here yet
    \end{itemize}
    \item Another fundamental issue is that support for sleepable \gls{bpf} is new and has
    not yet matured (only a small subset of \gls{lsm} programs can currently be marked
    sleepable)
    \item Linux 5.(11?) added \texttt{bpf\_d\_path}, but this is only callable from sleepable programs, a subset of \gls{lsm} programs
    \item In the current \bpfbox{} and \bpfcontain{} design, this limitation is too restrictive
    \item Luckily, it seems like the community is working towards a general solution to
    this problem (dynamic map allocation and making more program types sleepable)
    \item As the \gls{ebpf} ecosystem evolves, it may be possible to support pathnames as
    a first class citizen, removing the requirement for working with inodes and filesystem
    numbers
  \end{itemize}
\end{inprogress}

\subsection{Fixed-Size Policy Maps}

\begin{inprogress}
  \begin{itemize}
    \item Currently, policy maps are of a fixed size
    \item It's okay to make them big, since \gls{ebpf} does support map growth up to a fixed limit
    \item But we are still limited in total map size (todo: get current figure)
    \item In our current implementation, we simply grow policy maps from userspace when the map size would be too small to fit current rules
    \item But this approach is still limited, as it doesn't support map resizing at runtime, only at load time
    \item Once sleepable \gls{bpf} matures, we can have dynamically allocated maps of
    arbitrary size at runtime (link Alexei's LKML thread), as we can now have runtime map
    allocators where is it okay to sleep on a page fault
  \end{itemize}
\end{inprogress}

\subsection{Passive Performance Overhead}

\begin{inprogress}
  \begin{itemize}
    \item Currently, \bpfbox{} and \bpfcontain{} are competitive with mainstream
    confinement solutions based on \gls{lsm} (e.g.~AppArmor, see chapter 6)
    \item This competitiveness is actually an advantage, considering that \bpfbox{} and
    \bpfcontain{} can be dynamically loaded and attached to various system events. In this sense,
    we are getting increased flexibility without paying much of a cost in performance.
    \item But, \bpfbox{} and \bpfcontain{} have the potential to be far more performant than
    conventional \gls{lsm}s in one critical case: passive overhead on the rest of the system.
    \item Due to a current limitation of how KRSI works, its \gls{ebpf} \gls{lsm} hooks are
    always globally invoked, regardless of whether the target process is of interest to us or not.
    \item The current pattern looks like (Invoke Syscall $\rightarrow$ Invoke Hook
    $\rightarrow$ Invoke BPF Program $\rightarrow$ Filter Logic $\rightarrow$ Return from
    BPF Program $\rightarrow$ Return from Hook $\rightarrow$ Return from Syscall).
    \item In the future, we may be able to move the filter logic to the step
    \textit{before}  the hook is called, or at the very least before the BPF program is
    called. This would nearly eliminate any passive overhead on the unconfined parts of the system.
    \item I have a plan for this: introducing a new namespace for \gls{bpf} programs and
    maps. (Forward ref to BPF namespace and unprivileged BPF)
  \end{itemize}
\end{inprogress}

\subsection{Network Policy Granularity}

\begin{inprogress}
  \begin{itemize}
    \item Network policy in \bpfbox{} and \bpfcontain{} is currently very coarse-grained
    \item Only operates at the socket level, and does not considered nuanced access
    controls such as at the per-IP-address level.
    \item This means that specifying access to the network essentially gives the process or container
    access to the global network.
    \item While this is not problematic for applications that do not require network access,
    it quickly becomes an overprivilege issue for applications that do.
    \item To fix this, we can introduce a finer-grained network policy that specifies per-IP network access, an extension which
    is possible using currently available \gls{ebpf} technology. (Forward ref to protocol-level network policy)
  \end{itemize}
\end{inprogress}


\section{Future Work and Research Directions}%
\label{s:disc-future-work}

This section discusses opportunities for future work, both in terms of directly extending
\bpfbox{} and \bpfcontain{} and indirectly improving them through iterations on the
\gls{ebpf} ecosystem. We also discuss future research directions, proposing new evaluation
strategies and policy language experimentation that can further guide the development of
these research systems. In particular, we discuss conducting a user study to evaluate the
usability of \bpfbox{} and \bpfcontain{}, the addition of a new \gls{bpf} namespace and
unprivileged \gls{bpf}, integration with Docker and the \gls{oci} specification,
finer-grained network policy, automated policy generation, and a policy management \gls{gui}.

\subsection{Policy Language Experimentation and Usability Study}

\begin{inprogress}
  Policy language experimentation
  \begin{itemize}
    \item There is room for further policy language experimentation
    \item Perhaps \bpfbox{} would benefit from a similar policy language schema to \bpfcontain{}, rather than a DSL
    \item Or perhaps \bpfcontain{} would benefit from some alternate policy languages (different schemas, different serialization formats)
    \item Experiment with policy granularity, tunables that impact \bpfcontain{}'s default enforcement boundary
  \end{itemize}

  User study
  \begin{itemize}
    \item Determine quantitatively how \bpfcontain{}'s
    \item Introduce three research questions
    \begin{itemize}
      \item RQ1: How well do \bpfcontain{}'s policy semantics match user expectations?
      \begin{itemize}
        \item Does \bpfcontain{}'s default policy fit the user's mental model of a container? Any surprises?
        \item Is the \bpfcontain{} policy language suitable for the kind of confinement users want to do?
      \end{itemize}
      \item RQ2: How does the user experience of \bpfcontain{} compare with alternatives?
      \begin{itemize}
        \item Compare with alternative \glspl{lsm} (SELinux and AppArmor), seccomp-bpf, perhaps higher-level mechanisms like Snap
      \end{itemize}
      \item RQ3: How do users perceive the security of \bpfcontain{} compared with alternatives?
      \begin{itemize}
        \item Compare with alternative \glspl{lsm} (SELinux and AppArmor), seccomp-bpf, perhaps higher-level mechanisms like Snap
      \end{itemize}
    \end{itemize}
    \item Cite the FBAC-LSM paper for some ideas on how to conduct the user study
  \end{itemize}

  User study can inform further policy language experimentation
  \begin{itemize}
    \item Use answers to research questions to inform further design
    \item Adding new rule types, modifying existing rule types
    \item Change the design of the policy language, add more tunables, etc.
  \end{itemize}
\end{inprogress}

\subsection{The BPF Namespace and Unprivileged BPF}

\begin{inprogress}
  \begin{itemize}
    \item
  \end{itemize}
\end{inprogress}

\subsection{OCI and Docker Integration}

\begin{inprogress}
  \begin{itemize}
    \item
  \end{itemize}
\end{inprogress}

\subsection{Fine-Grained Network Policy}

\begin{inprogress}
  The problem with existing network policy
  \begin{itemize}
    \item \bpfbox{} and \bpfcontain{} currently support very coarse-grained networking policy
    \item They enforce network access control at the socket level; a process of container
    either has access to specific socket operations or they don't
    \item But this approach doesn't account for many of the nuances of network policy
    \item Allowing a process to use networking capability essentially opens it up to the outside world
    \item Further, it is often desirable to only have a container's network interface exposed to the local system
    \item Since policies lock down access to the rest of the system, this isn't as severe of a problem as it could be... Remote adversaries are still confined to accessing specific system resources according to the confinement policy
    \item But it would still be better to support much finer-grained networking policy from a least-privilege point of view
    \item Eliminate the need for a netfilter firewall (e.g.~iptables) for securing container's networking stack
  \end{itemize}

  How network policy could be extended
  \begin{itemize}
    \item \gls{ebpf} \texttt{TC\_CLSACT} program
    \item Stands for \enquote{traffic classifier and actions}
    \item Can be attached to a network interface, handles both ingress and egress
    \item Works similar in spirit to classic \gls{bpf} packet filters, but additionally supports making policy decisions and modifying packets
    \item We can use this program type, attached to a container's networking interface, to match specific characteristics of network traffic,
    such as IP address patterns, port numbers, and other protocol-level details
    \item The security policy itself cam be extended to support these characteristics in addition to basic socket operations
    \item We can then match this information against the container's security policy, and achieve much finer-grained network policy enforcement
  \end{itemize}
\end{inprogress}

\subsection{\bpfcontain{} Policy Generation and \glsentryshort{gui}}

\begin{inprogress}
  \begin{itemize}
    \item
  \end{itemize}
\end{inprogress}



\section{Conclusion}
\label{s:disc-conclusion}

This thesis has presented the design, implementation, and motivation behind \bpfbox{} and
\bpfcontain{}, two novel confinement mechanisms for the Linux kernel, with an emphasis on
simple yet precise policies, application- and container-specific confinement, and high
adoptability. We analyze the performance of these research systems and find that they are
competitive with existing confinement implementations while providing superior flexibility
for local confinement. While undoubtedly useful on their own, perhaps the greatest value
provided by these research systems is as a proof of concept, demonstrating the value of
\gls{ebpf}-based confinement solutions.

A myriad of potential extensions of \bpfbox{} and \bpfcontain{} can provide increased
security and flexibility as well as significantly reduce overhead to achieve better
performance than competitive solutions.  Extensions on top of the existing \gls{bpf}
ecosystem can help to position it as the dominant framework for implementing future
kernel-based security solutions. Such extensions can help to improve \bpfbox{} and
\bpfcontain{} as well as promote the adoption of other security mechanisms based on
\gls{ebpf}.

In the future, a security solution based on \gls{ebpf} may comprise a generic framework,
capable of loading and managing multiple \gls{bpf} program types to hook into any aspect
of system behaviour. Such a solution would encompass the capabilities of \bpfbox{} and
\bpfcontain{} and potentially expand them to include intrusion detection, network
filtering, software hot patches, and beyond. In the short term, improvements on top of the
\gls{ebpf} ecosystem as well as \bpfbox{} and \bpfcontain{} can make incremental progress
towards such a system.
