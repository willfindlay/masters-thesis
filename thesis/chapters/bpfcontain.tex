

\section{\bpfcontain{} Overview}

\todo{This section will present an architectural overview of \bpfcontain{}, summarize key
features and what it can do.}

\section{Extending \bpfbox{} to Model Containers}

\todo{This section will discuss how \bpfcontain{} extends \bpfbox{} to model containers.
Specifically, the idea is to enforce policy at the granularity of an entire container
rather than an individual process. This lets us get away with all sorts of implicit
policy\,---\,all operations within the confines of the container that do not affect the
rest of the system are permitted. Operations that impact the rest of the system, such as
those that modify kernel code, system parameters, or similar are denied. Everything else
can be specified as a rule. This basically allows us to get away with almost no policy
language whatsoever. A nice way to put it: \enquote{The policy language is used to define
the exceptions rather than the rules}.}



\section{\bpfcontain{} Implementation}

\todo{This section will present the implementation details of \bpfcontain{}, taken from our paper.}




\section{\bpfcontain{} Policy Language}

\todo{This section will present and document the policy language of \bpfcontain{}, taken from our paper.}




\section{Implicit Policy Under \bpfcontain{}}

\todo{This section will describe \bpfcontain{}'s implicit policy in more detail and describe the rationale behind the approach.}






\todo{How do we conclude this chapter?}
