In this chapter, we examine specific case studies, applying \bpfbox{} and \bpfcontain{}
policies to solve realistic problems. In particular, we examine the default Docker policy,
a more complex example involving a web server and database, and provide an example of how
\bpfcontain{} can be used to apply basic confinement to an untrusted container. To offer
a basis for comparison, we contrast presented policies with some available equivalents and
discuss how the semantics of the policy language and enforcement engine can impact the
resulting policy file.

% \section{Methodology}

% \todo{This section will present the methodology used to select existing policies and
% compare them with \bpfbox{} and \bpfcontain{} policies}


\section{The Default Docker Policy}

Docker~\cite{docker_security} applies a coarse-grained default confinement policy to all
containers using a combination of Linux confinement primitives. On supported
systems\footnote{Recall that not all Linux distributions support AppArmor or Seccomp-bpf
to begin with. In such cases, Docker simply discards its default confinement policy
altogether.}, this includes a default AppArmor policy template~\cite{docker_apparmor,
docker_default_apparmor}, a default Seccomp-bpf profile, and a set of POSIX capabilities
which are dropped at runtime~\cite{docker_security}.

Docker's policy defaults are highly coarse-grained, with an emphasis on practical security
while ensuring that the vast majority of container configurations will \enquote{just
work,} out of the box. This affords a practical opportunity to examine how \bpfbox{} and
\bpfcontain{} policies compare with the default Docker policy. \Cref{tab:docker-default}
summarizes the key aspects of Docker's confinement policy, highlighting default access
levels enforced by various Linux confinement primitives. \Cref{lst:docker-default} depicts
Docker's default AppArmor template, taken directly from the Docker sources on
GitHub~\cite{docker_default_apparmor}.

\begin{table}[htpb]
  \centering
  \caption[The default Docker confinement policy]{
    A summary of Docker's default confinement policy~\cite{docker_security,
    docker_apparmor, docker_default_apparmor}. Policy is enforced using a number of Linux
    confinement primitives, including AppArmor, Seccomp-bpf, and dropped POSIX
    capabilities at runtime. Docker generates and loads AppArmor policy at container
    runtime using a pre-determined, coarse-grained AppArmor template file
    (c.f.~\Cref{lst:docker-default}).
  }%
  \label{tab:docker-default}
  \footnotesize
  \begin{tabular}{lp{2in}p{1.6in}}
  \toprule
  Access Category & Default & Docker Implementation \\
  \midrule
  Files & Allow access to all files except specific procfs and sysfs entries. & AppArmor Template \\
  Filesystem Mounts & Deny all filesystem mounts. & AppArmor Template \\
  POSIX Capabilities & All capabilities enabled in AppArmor.  Drop specific capabilities at runtime. & AppArmor Template and Dropped Capabilities \\
  Ptrace & Allowed within container. & AppArmor Template \\
  Signals & Allowed within container. & AppArmor Template \\
  Network & Allow all network access. & AppArmor Template \\
  \gls{ipc} & Allow all \gls{ipc} access. & AppArmor Template \\
  System Calls & Deny about 60 obsolete/dangerous system calls. & Seccomp-bpf \\
  \bottomrule
  \end{tabular}
\end{table}

\begin{lstlisting}[language=none, gobble=4,
  caption={[Docker's default AppArmor template]
    Docker's default AppArmor template~\cite{docker_default_apparmor}, at the time of
    writing this thesis. Docker uses Go's string templating syntax to modify the AppArmor
    profile according to the current Docker version and container metadata.
  },
  label={lst:docker-default}, float]
    {{range $value := .Imports}}
      {{$value}}
    {{end}}
    profile {{.Name}} flags=(attach_disconnected,mediate_deleted) {
    {{range $value := .InnerImports}}
      {{$value}}
    {{end}}
      network,
      capability,
      file,
      umount,
    {{if ge .Version 208096}}
      # Host (privileged) processes may send signals to container processes.
      signal (receive) peer=unconfined,
      # dockerd may send signals to container processes (for "docker kill").
      signal (receive) peer={{.DaemonProfile}},
      # Container processes may send signals amongst themselves.
      signal (send,receive) peer={{.Name}},
    {{end}}
     # deny write for all files directly in /proc (not in a subdir)
      deny @{PROC}/* w,
      # deny write to files not in /proc/<number>/** or /proc/sys/**
      deny @{PROC}/{[^1-9],[^1-9][^0-9],
        [^1-9s][^0-9y][^0-9s],[^1-9][^0-9][^0-9][^0-9]*}/** w,
      # deny /proc/sys except /proc/sys/k* (effectively /proc/sys/kernel)
      deny @{PROC}/sys/[^k]** w,
      # deny everything except shm* in /proc/sys/kernel/
      deny @{PROC}/sys/kernel/{?,??,[^s][^h][^m]**} w,
      deny @{PROC}/sysrq-trigger rwklx,
      deny @{PROC}/kcore rwklx,
      deny mount,
      deny /sys/[^f]*/** wklx,
      deny /sys/f[^s]*/** wklx,
      deny /sys/fs/[^c]*/** wklx,
      deny /sys/fs/c[^g]*/** wklx,
      deny /sys/fs/cg[^r]*/** wklx,
      deny /sys/firmware/** rwklx,
      deny /sys/kernel/security/** rwklx,
    {{if ge .Version 208095}}
      # suppress ptrace denials when using 'docker ps' or using 'ps' inside a container
      ptrace (trace,read,tracedby,readby) peer={{.Name}},
    {{end}}
    }
\end{lstlisting}

We begin by examining a mostly equivalent policy in \bpfbox{}, given in
\Cref{lst:bpfbox-docker-default}.  Re-implementing Docker's default confinement policy in
\bpfbox{} is surprisingly challenging. \bpfbox{} is not designed to implement
coarse-grained confinement policy, and so specifying things like global access to all
files is impossible. We compromise by granting recursive access to all files within
a given filesystem, repeating the process for each filesystem as required. This is
\textit{not} the intended use case for \bpfbox{} file rules, but it is required to match
the over-permissive filesystem access provisioned by Docker. Aside from
filesystem-specific policy, most of Docker's default policy can be implemented relatively
easily and cleanly in \bpfbox{}'s policy language.

\begin{lstlisting}[language=bpfbox, gobble=4,
  caption={[Implementing the default Docker policy in \bpfbox{}]
    Implementing the default Docker policy in \bpfbox{}.
    \todo{High-level overview of the policy}
  },
  label={lst:bpfbox-docker-default}]
    #![profile "/path/to/init/program"]

    #[allow] {
      /* Allow essentially global access to a filesystem */
      fs("/path/to/filesystem/**", read|write|setattr|getattr|rm|link|ioctl)
      /* Repeat for others... */

      /* Allow access to /proc/sys/kernel/shm* */
      fs("/proc/sys/kernel/shm*", read|write|setattr|getattr)

      /* Sensible default access for procfs per-pid entries */
      proc(self, read|write)
      proc(child, read|write)
    }

    #[allow]
    #[taint]
    {
      /* Access to network families */
      net(inet, any)
      net(inet6, any)
      net(unix, any)
      /* Ptrace child processes */
      ptrace(child, read|write|attach)
      /* Send sigchld up to parent processes, any signal to children */
      signal(parent, sigchld)
      signal(child, any)
    }

    #[transition]
    #[untaint]
    {
      /* Allow execve calls to allowed executables,
       * tainting and transitioning profiles when doing so */
      fs("/path/to/allowed/executable", read|exec)
      /* Repeat for others... */
    }
\end{lstlisting}

Like Docker's AppArmor policy, our \bpfbox{} policy enables access to per-pid entries in
procfs and uses \bpfbox{}'s default-deny enforcement to restrict all others. Similar logic
applies to the \texttt{/proc/sys/kernel/shm**} entries under procfs. We also grant full
networking stack access, ptrace access for child processes, and full signal access for
child processes running under the container. Since these operations have the potential to
introduce vulnerabilities from outside sources, we mark them as tainting the corresponding
process. Leveraging taintedness, the \bpfbox{} policy eliminates the need to specify
access to shared library dependencies and other artifacts of the C runtime.

For more complex container deployments that include more than a single binary, the
\bpfbox{} policy may need to specify access to alternative executables under the
container.  We do so using an individual file rule for each executable, optionally
specifying that the process should untaint itself and/or transition to a new profile.
Notably, the version of \bpfbox{} presented in this thesis does \textit{not} include
capability-level policy, and so it is not included here\footnote{\bpfcontain{} later
rectified this gap in \bpfbox{}'s policy language.}. However, the default Docker
confinement policy does not implement capability-level filtering anyway, instead relying
on dropped capabilities at runtime.

Although the \bpfbox{} policy depicted in \Cref{lst:bpfbox-docker-default} does not fully
map to the precise Docker default policy, it gets very close in most respects, aside from
filesystem policy. Under \bpfbox{}, filesystem policy is necessarily finer-grained, as it
does not support the ability to specify coarse-grained access to all files on the system.
Despite these challenges, the end-result is a functional (and, in some aspects, more
secure) alternative to the default Docker policy.

\begin{lstlisting}[language=yaml, gobble=4,
  caption={[Implementing the default Docker policy in \bpfcontain{}]
    Implementing the default Docker policy in \bpfcontain{}. A few coarse-grained
    allow-rules can be used to capture permissive Docker defaults that are not covered
    under \bpfcontain{}'s default policy. Other aspects of the Docker defaults are already
    covered under \bpfcontain{} defaults, such as the inability to mount filesystems,
    perform a number of privileged system calls, and interact with non-pid entries in
    procfs and sysfs. Due to \bpfcontain{}'s default policy for file access and \gls{ipc},
    it is neither necessary to specify file access rules for files within the container's
    overlay filesystem not \gls{ipc} rules for processes within the container.
  },
  label={lst:bpfcontain-docker-default}]
    name: default-docker
    defaultTaint: false

    allow:
      # Grant access to the entire root filesystem
      fs: {pathname: /, access: any}
      # Grant access to the entire networking stack
      net: any
      # Enable Docker default capabilities
      # All other capabilities are denied
      capability:
        - chown
        - dacoverride
        - fsetid
        - fowner
        - mknod
        - netraw
        - setgid
        - setuid
        - setfcap
        - setpcap
        - netbindservice
        - syschroot
        - kill
        - auditwrite

    taint:
      net: any
\end{lstlisting}


\section{Confining a Web Server and Database}

\todo{This section will describe how \bpfbox{} and \bpfcontain{} can be used to confine an
Apache Webserver and Postgres Database configuration and then compare the resulting policy
with Snap, AppArmor, and SELinux}

\section{The Untrusted Container Use Case}

\todo{\bpfbox{} is not the right tool for this job, but \bpfcontain{} is}

\begin{lstlisting}[language=yaml, gobble=4,
  caption={[Confining an untrusted container with \bpfcontain{}]
    Confining an untrusted container with \bpfcontain{}. \bpfcontain{}'s default
    enforcement policy of defining a boundary around the container enables this policy to
    be quite simple. A default-tainted policy enables container-level confinement without
    specifying \textit{any} rules whatsoever. This policy can then be adjusted as
    required, specifying file rules to provision access to volume mounts, network rules to
    enable networking, and capability rules to enable access to specific POSIX
    capabilities.
  },
  label={lst:bpfcontain-untrusted}, float]
    name: untrusted-container
    defaultTaint: true

    allow:
      # Specify full path and access for volume mounts from the host
      file: {pathname: /path/to/volume/mount, access: rw}
      # Uncomment if the container requires networking
      # net: any
      # Uncomment if the container requires any capabilities
      # capability: [capDacOverrride] # etc.
\end{lstlisting}

