\begin{inprogress}
  \begin{itemize}
    \item Might be useful: \cite{dranger2006_dac_complexity}
  \end{itemize}

Researchers have been studying confinement for decades~\cite{lampson1973_confinement}, and
have been designing and applying confinement primitives since the early days of
time-sharing computers and multi-tenant systems~\cite{shu2016_security_isolation_study}.
Process-level virtualization has become similarly important, but for subtly different
reasons\,---\,namely, for the purpose of restricting and modifying a process' visibility
into the rest of the system.  Decades of research into virtualization and confinement
techniques has culminated in a wealth of knowledge and prior artifacts, some of which have
seen widespread deployment and remain relevant to date. The rest of this section presents
technical background on such process-level virtualization confinement mechanisms and
surveys the academic literature related to this area.
\end{inprogress}


\begin{inprogress}
  Container security policies, where they even exist, are often overly-generic and
  ill-suited to fine-grained confinement. To achieve confinement in the first place,
  container frameworks cobble together existing confinement technologies and apply them in
  confusing and difficult-to-audit ways, overlapping and recombining default and generated
  policies. The end result is a complex security soup with little room for policy
  customization or auditability. \todo{Examples here}

  Even worse, many container management systems operate under a fail-open approach when the
  necessary security mechanisms are not supported. This results in low-security deployments,
  often without even notifying the user that there may be such a configuration. Since the
  end-user generally doesn't even participate in the policy authorship process, they may not
  even be aware of the level of protection that is being applied, resulting in a dangerous
  false sense of security. \todo{Examples here}
\end{inprogress}
