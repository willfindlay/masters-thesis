Virtualization is not confinement. To security experts, this may be an obvious statement,
but these two concepts are often conflated, leading to dangerous assumptions about
security in practice. To see why virtualization and confinement are disparate concepts,
consider the goals of each. \textit{Virtualization} describes the goal of providing
a unique, private mapping of shared system resources to a particular subject~\todo{CITE}
(be it a process, a virtualized operating system, or something else).
\textit{Confinement}, on the other hand, describes the goal of restricting a subjects
access to system resources or other subjects~\todo{CITE}. In other words, virtualization
is about \textit{what we can see}, whereas confinement is about \textit{what we can do}.
It should now be abundantly clear not only that virtualization and confinement are two
entirely different concepts, but that virtualization must be combined with confinement to
offer any practical security guarantees.

Container technologies on Linux offer a motivating example of the difference between
virtualization and confinement and how conflating the two can result in problematic
misunderstandings about the security of a system. From first principles, a container is
nothing more than a group of associated processes (and system resources) managed by some
container management engine (e.g.~Docker or \textit{lxc}). While often used for dependency
management, containers also offer lightweight virtualization and confinement using
a series primitives exposed by the operating system kernel. Namespaces and cgroups
virtualize system resources while confinement layers such as \texttt{seccomp(2)} provide
some degree of isolation from the rest of the system~\todo{CITE}. Unfortunately, while
virtualization primitives are widely used in container deployments, the use of confinement
primitives is often overlooked, oversimplified, or overly permissive (i.e.~misconfigured)
in practice~\todo{CITE}.

Despite their name, containers do a very poor job of actually \textit{containing}
a running process.  The Linux kernel already supports a set of strong virtualization
primitives, namespaces and cgroups, that can be used to provide a process with a fairly
compartmentalized view of its environment. But these virtualization primitives do little
to improve the \textit{security} of the applications that use them. Rather, such
virtualization serves much the same purpose as virtualization in other areas of computer
science\,---\,to allow two pieces of code to use the same name for different things and to
abstract away the complex details of the underlying system.

This thesis argues that the key to truly isolating containers from the rest of the system
lies in improving our confinement frameworks to work well in containerized applications.
The status quo of process confinement on Linux is a motley collection of unrelated
technologies, each designed for a specific purpose, often beyond the scope of simple
process confinement. Container management solutions cobble these technologies together,
applying over-permissive, coarse-grained defaults or compiling simple policies into
a complex entanglement. To rectify the status quo, we must rethink confinement from the
ground up, and build a confinement framework designed to be aware of the container model
of computation.

In particular, we posit that the root of the confinement problem in container security
lies in a semantic gap between the containers themselves and the confinement primitives
that are used to secure them. Many academics have recognized this semantic gap. Some have
attempted to rectify it by introducing mechanisms for automatic policy derivation based on
container images (e.g.~the Docker image manifest) or observed behaviours at
runtime~\todo{CITE ALL EXAMPLES}. Others have \todo{what did others do?}. We propose that
the heart of the problem lies in the fact that the Linux kernel itself has no single
representation of \textit{what exactly a container is} from a security standpoint.

One approach would be to extend the container with a container-aware LSM, but maintaining
out-of-tree kernel modules is challenging, as such a module would need to be continually
updated as the kernel evolves. Further, end-users may be reluctant to adopt and
out-of-tree solution, particularly in production environments with low risk-tolerance.
Even worse, a static security mechanism is unlikely to work well for all use
cases\,---\,some users (e.g.~operators of multi-tenant container clouds) would like to
totally isolate containers from each other, while use cases (e.g.~microservices
deployments) may require individual containers to communicate and cooperate with each
other. Thus, our solution should be practical, light-weight, and flexible such that we can
define a container-level policy enforcement mechanism that can be readily adopted in
production use cases. A relatively new Linux technology, \textit{Extended Berkley Packet
Filter} (eBPF), offers the opportunity to design such a system.

To improve the status quo of confinement on Linux, we present two research prototypes,
\bpfbox{} and its successor, \bpfcontain{}. The former is a novel application sandboxing
framework, and the latter extends that framework to work well in the context of container
security. Both research systems are implemented using eBPF, a new Linux kernel mechanism
for dynamically attaching simple filter programs to various system events and aggregating
data from these events in kernelspace~\cite{gregg2019_bpf, starovoitov2014_ebpf}.
Specifically, we use the new LSM program type~\cite{singh2019_krsi}, introduced in Linux
5.7, to attach eBPF programs to Linux Security Module hooks. Using eBPF, we can safely
extend the kernel at runtime, building a new confinement model that suits the container
use case without detracting from existing confinement implementations or tying the kernel
down to one particular model.


\section{Motivation}%
\label{s:motivation}

\subsection{Contextualizing the Problem}%
\label{ss:contextualizing-the-problem}

Containers are \textit{everywhere}. In the cloud, containers form the backbone of
cloud-native computation. Kubernetes~\todo{CITE} clusters drive the microservices that
power scalable web applications. In devops, Docker~\todo{CITE} containers often form the
backbone of continuous integration workflows, providing reproducible environments for
development, testing, and debugging. On the desktop, containerized package managers like
Snap~\todo{CITE}, FlatPak~\todo{CITE}, and AppImage~\todo{CITE} offer self-contained,
isolated software bundles, facilitating a smooth software installation process (mostly)
free of dependency management concerns.

Despite a steadily increasing prevalence, containers face major adoptability challenges in
deployments where they are expected to outright replace virtual machines. Unlike virtual
machines, which are abstracted away from the host and interact with a hypervisor,
containers interact directly with the host operating system kernel. This means that, while
much lighter-weight than hypervisor-based virtualization, containers are inherently less
isolated from each other and from the host system in
general~\cite{sultan2019_container_security, xin2018_container_security,
mullinix2020_security_measures, bui2015_docker_analysis}.  In order to have truly secure
containers, we must take great care to ensure that a container is properly
\textit{confined}. In practice, this means restricting the processes that run within the
container from performing certain actions that can negatively impact or damage the system.
As we have already discussed, virtualization primitives alone are not enough to achieve
proper isolation. These primitives \textit{must} be combined with confinement mechanisms
and these confinement mechanisms \textit{must} be applied properly. Otherwise, we risk
overprivilege, resulting in potential violations of our security model.

Container security issues are widely studied in the
literature~\cite{sultan2019_container_security, xin2018_container_security,
mp2016_hardening, mullinix2020_security_measures, bui2015_docker_analysis}.  Given that
containers offer weaker isolation guarantees than alternatives like hardware virtual
machines (HVMs)~\todo{CITE} or even paravirtualization~\todo{CITE}, one might be inclined
to assume that container security would be of paramount importance. Contrary to this
assumption, container management frameworks take a lax attitude towards the enforcement of
least-privilege, provisioning overly-permissive blanket default policies and relying on
a complex suite of ill-suited security mechanisms provided by the host system.

Docker, for instance, applies a default AppArmor policy revoking access to only the most
sensitive kernel interfaces like procfs and sysfs and disabling the ability to mount new
filesystem.  Beyond these basic controls, the container has full permission to access all
filesystem and resources, has access to several POSIX capabilities, and may unmount any
filesystem~\cite{docker_apparmor, docker_default_apparmor}. Even worse, a kernel that does
not support AppArmor or that is not properly configured is left totally bereft of this
protection to begin with.  Docker complements its default AppArmor profile with a set of
sensible seccomp rules, revoking access to many privileged system calls. While such
a policy \textit{does} help to harden the container, it remains
overly-generalized~\cite{sultan2019_container_security} and does not uniquely capture the
needs of every container deployment. Users who wish to grant additional permissions to
their container are left with the choice of either writing and auditing custom AppArmor
and seccomp policies or outright disabling protections altogether with the
\texttt{--privileged} flag.

Docker is but the most prominent example among many. In general, all existing container
management frameworks rely on a patchwork of isolation mechanisms, each enforcing its own
confinement policy and each with varying degrees of generalization. As a result, these
policies are often difficult to reason about, and thus difficult to effectively audit.
A vulnerability in any individual mechanism or a misconfiguration in any individual policy
opens the container or the host system itself up to attack. Blanket defaults are often
ineffective for specific use cases and result in situations where the end-user is forced to
either abandon all hope of security or muddle through the configuration of multiple policy
enforcement mechanisms.

\subsection{Why Design a New Confinement Framework?}%
\label{ss:why-new}

The process confinement problem dates back half a century~\cite{lampson1973_confinement}.
Since the advent of multi-processing \todo{IN WHAT YEAR? (e.g.~with Unix in 196X) CITE},
security experts have been concerned with designing systems in such a way that two running
programs minimally interfere with one another. Since then, an abundance of tools and
frameworks, some more practical than others, have been proposed to limit the damage that
untrusted software can do to the system as a whole~\cite{shu2016_security_isolation_study}.
These are covered in more depth in \Cref{c:background}. For now, we focus on why it might be
prudent to design yet another confinement framework amidst this veritable ocean of prior work.

The Linux kernel already provides a number of confinement primitives. Seccomp allows for
a process to confine itself by filtering the system calls it can make. Mandatory access
control solutions based on LSM (Linux Security Modules) hooks can be configured to define
and enforce powerful per-application policies, protecting system resources from unwanted
access. Unix DAC (discretionary access control) restricts access to system resources
according to resource owners, groups, permission bits, and access control lists. When
applied to container security, the common problem faced by these security mechanisms is
that they are being applied to solve a problem for which they were not originally
designed. To solve this problem, we seek to design a unifying security abstraction for
containers and apply this abstraction to enforce per-container policy in kernelspace.

From the kernel's perspective, a containerized process is just like any other. While it
may be virtualized under one or more namespace and process control groups, there is no
precise definition of what exactly constitutes a \textit{container}. This lack of a solid
abstraction widens the semantic gap between traditional policy enforcement mechanisms and
security policy designed to protect containers. In defining a new policy enforcement mechanism
focused specifically on containers, we have an opportunity to narrow this semantic gap,
simplify the resulting policies, and eliminate the need to combine several security mechanisms
together to do a job that could be accomplished by just one. Since our proposed solution is based
on eBPF, it requires no modification of the kernel and can be dynamically loaded at runtime.
This means that we can provide such a unified abstraction without sacrificing forward
or backward compatibility with alternative approaches.

\subsection{Why eBPF?}%
\label{ss:why-ebpf}

An eBPF-based confinement mechanism provides several advantages over traditional
confinement models.  Firstly, eBPF is \textit{light-weight}. eBPF programs can monitor
many aspects of system behaviour, from userspace function calls to kernelspace function
calls, system calls, security hooks, and the networking stack. Data from these events can
be aggregated in real time in kernelspace, providing an extensible, performant, and
flexible framework for modelling relationships and enforcing policy decisions based on
these relationships.  A single security mechanism based on eBPF can combine the advantages
of several disparate mechanisms that would ordinarily need to be combined together to
provide full protection. This property precisely mirrors the way container security is
currently done on Linux. Rather than combining namespaces, cgroups, seccomp, and mandatory
access control together, eBPF provides the opportunity to design a single framework
providing the advantages of each.

A second advantage of eBPF for writing a security framework is that it is
\textit{dynamic}. eBPF programs can be loaded into the kernel dynamically and attached to
multiple events. Instrumenting a system event with eBPF can be done at runtime,
\textit{without} the need to modify the kernel in any way.  Similarly, eBPF maps, the
canonical runtime data store for eBPF programs, can be loaded, unloaded, modified, and
queried at runtime from both userspace and kernelspace, providing a rich substrate for
a dynamic model of system behaviour. These properties culminate in the ability to design
a flexible security mechanism without tying the kernel down to any one particular
abstraction. In the context of container security, this is a particularly important goal,
as containers are traditionally a \textit{userspace} concept, glued together with various
abstractions provided by the kernel.

\textit{Production-safety} is a third advantage provided by eBPF. All eBPF programs go
through a verification process before they are loaded into the kernel. The eBPF verifier
analyzes the program, asserting that it conforms to a number of safety requirements, such
as program termination\footnote{This property is enforceable due to the fact that eBPF
programs are not Turing complete~\cite{gregg2019_bpf}.}, memory safety, and read-only
access to kernel data structures. While itself not formally verified, the eBPF verifier
facilitates the adoption of new eBPF programs into production use cases, since an eBPF
program is far less likely to adversely impact a production system than other methods of
extending the kernel (e.g.~kernel patches and loadable kernel modules). In fact, eBPF is
already being used in production at large datacenters by Facebook, Netflix, Google, and
others to monitor server workloads for security and performance
regressions~\cite{gregg2019_bpf}. These factors make eBPF a promising choice for
designing an \textit{adoptable} security mechanism.

In summary, eBPF offers unique and promising advantages for developing novel security
mechanisms. Its light-weight execution model coupled with the flexibility to monitor and
aggregate events across userspace and kernelspace provide the ability to control and audit
nearly any aspect of the running system. eBPF maps, shareable across programs and between
userspace and the kernel offer a means of aggregating data from multiple sources at
runtime and using it to inform policy decisions across domains. A security mechanism based
on eBPF can be dynamically loaded into the kernel as needed, and eBPF's safety guarantees
combined with its increasing adoption in production use cases provide strong adoptability
advantages. This means that a security mechanism based on eBPF can be both adoptable and
effective.


\section{Contributions}%
\label{s:contributions}


\section{Outline}%
\label{s:outline}
