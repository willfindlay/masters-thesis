Virtualization is not confinement. To security experts, this may be an obvious statement,
but these two concepts are often conflated, leading to dangerous assumptions about
security in practice. To see why virtualization and confinement are disparate concepts,
consider the goals of each. \textit{Virtualization} describes the goal of providing
a unique, private mapping of shared system resources to a particular subject~\todo{CITE}
(be it a process, a virtualized operating system, or something else).
\textit{Confinement}, on the other hand, describes the goal of restricting a subjects
access to system resources or other subjects~\todo{CITE}. In other words, virtualization
is about \textit{what we can see}, whereas confinement is about \textit{what we can do}.
It should now be abundantly clear that virtualization and confinement are not only two
entirely different concepts, but that virtualization must be combined with confinement to
offer any practical security guarantees.

\begin{inprogress}
  Container technologies on Linux offer a motivating example of the difference between
  virtualization and confinement and how conflating the two can result in problematic
  misunderstandings about the security of a system. Linux containers offer lightweight
  virtualization and confinement using a series primitives exposed by the operating system
  kernel. Namespaces and cgroups virtualize system resources while confinement layers such
  as \texttt{seccomp(2)} provide some degree of isolation from the rest of the system~\todo{CITE}.
  Unfortunately, while virtualization primitives are widely used in container deployments,
  the use of confinement primitives is often overlooked, oversimplified, or overly
  permissive (i.e.~misconfigured)~\todo{CITE}.
\end{inprogress}


\section{Contributions}%
\label{s:contributions}
