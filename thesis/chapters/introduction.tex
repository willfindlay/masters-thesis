% Virtualization is not confinement. To security experts, this may be an obvious statement,
% but these two concepts are often conflated, leading to dangerous assumptions about
% security in practice. To see why virtualization and confinement are disparate concepts,
% consider the goals of each. \textit{Virtualization} describes the goal of providing
% a unique, private mapping of shared system resources to a particular subject~\todo{CITE}
% (be it a process, a virtualized \gls{os}, or something else).
% \textit{Confinement}, on the other hand, describes the goal of restricting a subject's
% access to system resources or other subjects~\todo{CITE}. In other words, virtualization
% is about \textit{what we can see}, whereas confinement is about \textit{what we can do}.
% It should now be abundantly clear not only that virtualization and confinement are two
% entirely different concepts, but that virtualization must be combined with confinement to
% offer any practical security guarantees.

% Container technologies on Linux offer a motivating example of the difference between
% virtualization and confinement and how conflating the two can result in problematic
% misunderstandings about the security of a system. From first principles, a container is
% nothing more than a group of associated processes (and system resources) managed by some
% container management engine (e.g.\ Docker or \textit{lxc}). While often used for dependency
% management, containers also offer lightweight virtualization and confinement using
% a series primitives exposed by the \gls{os} kernel. Namespaces and cgroups virtualize
% system resources while confinement layers such as \texttt{seccomp(2)} provide some degree
% of isolation from the rest of the system~\cite{sultan2019_container_security}.
% Unfortunately, while virtualization primitives are widely used in container deployments,
% the use of confinement primitives is often overlooked, oversimplified, or overly
% permissive (i.e.\ misconfigured) in practice~\cite{sultan2019_container_security}.

% Despite their name, containers do a very poor job of actually \textit{containing}
% a running process.  The Linux kernel already supports a set of strong virtualization
% primitives, namespaces and cgroups, that can be used to provide a process with a fairly
% compartmentalized view of its environment. But these virtualization primitives do little
% to improve the \textit{security} of the applications that use them. Rather, such
% virtualization serves much the same purpose as virtualization in other areas of computer
% science\,---\,to allow two pieces of code to use the same name for different things and to
% abstract away the complex details of the underlying system.

% This thesis argues that the key to truly isolating containers from the rest of the system
% lies in improving our confinement frameworks to work well in containerized applications.
% The status quo of process confinement on Linux is a motley collection of unrelated
% technologies, each designed for a specific purpose, often beyond the scope of simple
% process confinement. Container management solutions cobble these technologies together,
% applying over-permissive, coarse-grained defaults or compiling simple policies into
% a complex entanglement. To rectify the status quo, we must rethink confinement from the
% ground up, and build a confinement framework designed to be aware of the container model
% of computation.

% In particular, we posit that the root of the confinement problem in container security
% lies in a semantic gap between the containers themselves and the confinement primitives
% that are used to secure them. Many academics have recognized this semantic gap. Some have
% attempted to rectify it by introducing mechanisms for automatic policy derivation based on
% container images (e.g.\ the Docker image manifest) or observed behaviours at
% runtime~\todo{CITE ALL EXAMPLES}. Others have attempted to introduce new layers of
% abstraction to security policies~\todo{CITE Snap et al} or into the kernel
% itself~\todo{CITE security namespaces}. \todo{WHY ARE THESE BAD IN ONE SENTENCE?} This
% thesis proposes that the heart of the problem lies in the fact that the Linux kernel
% itself has no single representation of \textit{what exactly a container is} from
% a security standpoint.

% One approach would be to extend the container with a container-aware \gls{lsm}, but
% maintaining out-of-tree kernel modules is challenging, as such a module would need to be
% continually updated as the kernel evolves. Further, end users may be reluctant to adopt
% and out-of-tree solution, particularly in production environments with low risk-tolerance.
% Even worse, a static security mechanism is unlikely to work well for all use
% cases\,---\,some users (e.g.\ operators of multi-tenant container clouds) would like to
% totally isolate containers from each other, while use cases (e.g.\ microservices
% deployments) may require individual containers to communicate and cooperate with each
% other. Thus, our solution should be practical, lightweight, and flexible such that we can
% define a container-level policy enforcement mechanism that can be readily adopted in
% production use cases. A relatively new Linux technology,
% \gls{ebpf}~\cite{starovoitov2014_ebpf, gregg2019_bpf}, offers the opportunity to design
% such a system.

Virtualization is not confinement. Put simply, that which we can \textit{see} is not the
same thing as that which we can \textit{do}. To security experts, this may be an obvious
statement; however, not every user of an information technology system is a security
expert, nor should they be. Unfortunately, these two disparate yet related concepts are
often conflated, leading to dangerous security assumptions in practice. In particular, we
tend to assume that a virtualized system is the same thing as a secure system, which may
not necessarily be the case. Confinement is critically important to maintaining the
\textit{principle of least-privilege}~\cite{schneider03_least_privilege}, a quintessential
property of a truly secure system~\cite{van_oorschot2020_tools_jewels}. Despite playing
a critical role in systems security, the state of confinement on Linux is ill-suited to
meet the practical needs end users.

Existing Linux confinement mechanisms are complex, and often target specific use cases
beyond simple confinement. This results in a motley collection of isolation primitives
being used to lock down basic application functionality. Namespaces and cgroups virtualize
system resources while Linux security modules, \texttt{seccomp(2)}, discretionary access
controls, and more are used to restrict access. The need to combine these mechanisms
begets unintuitive inter-dependence relationships that lead to additional complexity and
security policies that are both painful to write and difficult to audit. In turn, these
difficulties weaken the ultimate authority of the system administrator, shifting the
burden of confinement onto distribution vendors or application authors.

Linux containers~\cite{docker_security, lxc_security, lin2018_container_security,
sultan2019_container_security} are a motivating example of this phenomenon. Intuitively,
a user might be motivated to use a container to \textit{contain things}. The reality of
container confinement does not match this intuition. Containers are nothing more than
a group of related processes (or even a single process) united by a shared view of system
resources (based on Linux virtualization primitives). Despite their name, containers in
general do a very poor job of actually containing anything. In particular, security
defaults in container management engines like Docker~\cite{docker_security} or
\gls{lxc}~\cite{lxc_security} rely on a myriad of unrelated confinement primitives, many
of which were designed to holistically lock down entire systems. Misuse of these
primitives results in a complex entanglement of related policies that ultimately must be
simplified down to their basest elements. The result is containers that \enquote{just
work}, albeit operating under highly coarse-grained policies that provide little
protection in practice~\cite{sultan2019_container_security, lin2018_container_security}.

This thesis argues that the key to implementing lightweight confinement policies that
work well in the context of containers lies in simplifying and unifying the underlying
confinement framework, and bridging the semantic gap between confinement policies and the
applications or containers they are designed to protect. In the past, this may have been
a difficult problem to solve. After all, existing confinement mechanisms are designed for
general-purpose use cases, and the precise definition of what constitutes
\enquote{container semantics} varies depending on the needs of the container deployment
and the design of the container management engine. We posit that the key to designing
a confinement mechanism that meets these goals is the ability to dynamically modify and
extend the kernel's security monitor, building a security framework that is easy to deploy
and simple to extend and modify. A new Linux technology, \gls{ebpf}, now enables us to
fill this technological gap.

Specifically, \gls{ebpf}~\cite{gregg2019_bpf, starovoitov2014_ebpf} enables a privileged
userspace process to safely and dynamically add simple hooking and filtering logic to key
components of the kernel. By designing and deploying a specific set of \gls{ebpf}
programs, we can adjust the kernel's security semantics, without necessarily tying it down
to a specific \enquote{one-size-fits-all} confinement solution. This enables us to build
application- or container-specific policies that scale well and meet the needs of
end users.

To improve the status quo of confinement on Linux, we present two research prototypes,
\bpfbox{} and its successor \bpfcontain{}. The former is a novel application sandboxing
framework, and the latter extends that framework to work well in the context of container
security. Both research systems are implemented using \gls{ebpf}, the first such systems
of their kind. In this thesis, we present a motivating re-framing of the confinement
problem, examine the design and implementation of \bpfbox{} and \bpfcontain{}, and show
that they improve application and container security without a significant impact on
system performance.

% Specifically, we use the new \gls{lsm} program type~\cite{singh2019_krsi},
% introduced in Linux 5.7, to attach \gls{ebpf} programs to Linux Security Module hooks.
% Using \gls{ebpf}, we can safely extend the kernel at runtime, building a new confinement
% model that suits the container use case without detracting from existing confinement
% implementations or tying the kernel down to one particular model.

\section{Research Questions}%
\label{s:intro-rqs}

In this thesis, we consider the following research questions.

\begin{rqenum}
  \item \label{rq1} What difficulties in the current state of Linux confinement lead to the semantic
        gaps between policies and the entities they are designed to lock down? How can we design
        a novel confinement mechanism to rectify these difficulties?

  \item \label{rq2} Can \gls{ebpf} be used to implement a full-featured confinement framework?
        What would such a framework look like and how could it be made to model container semantics?

  \item \label{rq3} What levels of security and performance can we expect from a confinement mechanism
        designed around \gls{ebpf}? What improvements to \gls{ebpf} would be required for
        a complete solution?
\end{rqenum}

\section{Motivation}%
\label{s:motivation}

\subsection{Contextualizing the Problem}%
\label{ss:contextualizing-the-problem}

Containers are \textit{everywhere}. In the cloud, containers form the backbone of
cloud-native computation. Kubernetes~\cite{kubernetes} clusters drive the microservices
that power scalable web applications. In devops, Docker~\cite{docker_security} containers
often form the backbone of continuous integration workflows, providing reproducible
environments for development, testing, and debugging. On the desktop, containerized
package managers like Snap~\cite{snap} and FlatPak~\cite{flatpak} offer self-contained,
isolated software bundles, facilitating a smooth software installation process (mostly)
free of dependency management concerns.

Despite a steadily increasing prevalence, containers face major adoptability challenges in
deployments\footnote{E.g., Cloud-Native deployments~\cite{brady2020_docker_cloud}.} where
they are expected to outright replace virtual machines. Unlike virtual machines, which are
abstracted away from the host and interact with a hypervisor, containers interact directly
with the host operating system kernel. This means that, while much more lightweight than
hypervisor-based virtualization, containers are inherently less isolated from each other
and from the host system in general~\cite{sultan2019_container_security,
lin2018_container_security, mullinix2020_security_measures, bui2015_docker_analysis}.  In
order to have truly secure containers, we must take great care to ensure that a container
is properly \textit{confined}. In practice, this means restricting the processes that run
within the container from performing certain actions that can negatively impact or damage
the system.  As we have already discussed, virtualization primitives alone are not enough
to achieve proper isolation. These primitives \textit{must} be combined with confinement
mechanisms and these confinement mechanisms \textit{must} be applied properly. Otherwise,
we risk overprivilege, resulting in potential violations of our security model.

Container security issues are widely studied in the
literature~\cite{sultan2019_container_security, lin2018_container_security,
mp2016_hardening, mullinix2020_security_measures, bui2015_docker_analysis}.  Despite the
fact that containers run directly on the host operating system and share a single kernel
with other native processes, security is generally treated as an afterthought in the
design of container management engines. If we truly want containers to be as secure as
virtual machines, we must rethink the way we secure container deployments. Security must
be prioritized from the ground up but must not get in the way of functionality. Existing
container frameworks accomplish the second goal but not the first.

Docker, for instance, applies a default AppArmor policy revoking access to only the most
sensitive kernel interfaces like procfs and sysfs and disabling the ability to mount new
filesystems. Beyond these basic controls, the container has full permission to access all
filesystem resources, has access to several POSIX capabilities, and may unmount any
filesystem~\cite{docker_apparmor, docker_default_apparmor}. Even worse, a kernel that does
not support AppArmor or that is not properly configured is left totally bereft of this
protection to begin with. Docker complements its default AppArmor profile with a set of
sensible seccomp rules, revoking access to many privileged system calls. While such
a policy \textit{does} help to harden the container, it remains
overly-generalized~\cite{sultan2019_container_security} and does not uniquely capture the
needs of every container deployment. Users who wish to grant additional permissions to
their container are left with the choice of either writing and auditing custom AppArmor
and seccomp policies or outright disabling protections altogether with the
\texttt{--privileged} flag.

Docker is but the most prominent example among many. In general, all existing container
management frameworks rely on a patchwork of isolation mechanisms, each enforcing its own
confinement policy and each with varying degrees of generalization. As a result, these
policies are often difficult to reason about, and thus are difficult to effectively audit.
A vulnerability in any individual mechanism or a misconfiguration in any individual policy
opens the container or the host system itself up to attack. Blanket defaults are often
ineffective for specific use cases and result in situations where the end user is forced to
either abandon all hope of security or muddle through the configuration of multiple policy
enforcement mechanisms.

\subsection{Why Design a New Confinement Framework?}%
\label{ss:why-new}

The process confinement problem dates back half a century~\cite{lampson1973_confinement}.
Since the advent of multi-processing and multi-tenant systems in the 1960s and
1970s~\cite{vyssotsky1965_multics, corbato1965_multics, ritchie1974_unix} with Multics and
Unix, security experts have been concerned with designing systems in such a way that two
running programs minimally interfere with one another. Since then, an abundance of tools
and frameworks, some more practical than others, have been proposed to limit the damage
that untrusted software can do to the system as
a whole~\cite{shu2016_security_isolation_study}. These are covered in more depth in
\Cref{c:background}. For now, we focus on why it might be prudent to design yet another
confinement framework amidst this veritable ocean of prior work.

The Linux kernel already provides a number of confinement primitives. Seccomp allows for
a process to confine itself by filtering the system calls it can make. Mandatory access
control solutions based on \gls{lsm} hooks can be configured to define and enforce
powerful per-application policies, protecting system resources from unwanted access. Unix
\gls{dac}~\cite{ritchie1974_unix, van_oorschot2020_tools_jewels, jaeger2008_os_security,
shu2016_security_isolation_study} restricts access to system resources according to
resource owners, groups, permission bits, and access control lists. When applied to
container security, the common problem faced by these security mechanisms is that they are
being applied to solve a problem for which they were not originally designed. To solve
this problem, we seek to design a unifying security abstraction for containers and apply
this abstraction to enforce per-container policy in kernelspace.

From the kernel's perspective, a containerized process is just like any
other~\cite{sultan2019_container_security}. While it may be virtualized under one or more
namespace and process control groups, there is no precise definition of what exactly
constitutes a \textit{container}. This lack of a solid abstraction widens the semantic gap
between traditional policy enforcement mechanisms and security policy designed to protect
containers. In defining a new policy enforcement mechanism focused specifically on
containers, we have an opportunity to narrow this semantic gap, simplify the resulting
policies, and eliminate the need to combine several security mechanisms together to do
a job that could be accomplished by just one. Since our proposed solution is based on
\gls{ebpf}, it requires no modification of the kernel and can be dynamically loaded at
runtime.  This means that we can provide such a unified abstraction without sacrificing
forward or backward compatibility with alternative approaches.

\subsection{Why eBPF?}%
\label{ss:why-ebpf}

An \gls{ebpf}-based confinement mechanism provides several advantages over traditional
confinement models.  Firstly, \gls{ebpf} is \textit{lightweight}. \gls{ebpf} programs can monitor
many aspects of system behaviour, from userspace function calls to kernelspace function
calls, system calls, security hooks, and the networking stack. Data from these events can
be aggregated in real time in kernelspace, providing an extensible, performant, and
flexible framework for modelling relationships and enforcing policy decisions based on
these relationships.  A single security mechanism based on \gls{ebpf} can combine the advantages
of several disparate mechanisms that would ordinarily need to be combined together to
provide full protection. This notion is the antithesis of the way container security is
currently done on Linux. Rather than combining namespaces, cgroups, seccomp, and mandatory
access control together, \gls{ebpf} provides the opportunity to design a single framework
providing the advantages of each.

A second advantage of \gls{ebpf} for writing a security framework is that it is
\textit{dynamic}. \gls{ebpf} programs can be loaded into the kernel dynamically and attached to
multiple events. Instrumenting a system event with \gls{ebpf} can be done at runtime,
\textit{without} the need to modify the kernel in any way.  Similarly, \gls{ebpf} maps, the
canonical runtime data store for \gls{ebpf} programs, can be loaded, unloaded, modified, and
queried at runtime from both userspace and kernelspace, providing a rich substrate for
a dynamic model of system behaviour. These properties culminate in the ability to design
a flexible security mechanism without tying the kernel down to any one particular
abstraction. In the context of container security, this is a particularly important goal,
as containers are traditionally a \textit{userspace} concept, glued together with various
abstractions provided by the kernel.

\textit{Production safety} is a third advantage provided by \gls{ebpf}. All \gls{ebpf} programs go
through a verification process before they are loaded into the kernel. The \gls{ebpf} verifier
analyzes the program, asserting that it conforms to a number of safety requirements, such
as program termination\footnote{This property is enforceable due to the fact that \gls{ebpf}
programs are not Turing complete~\cite{gregg2019_bpf}.}, memory safety, and read-only
access to kernel data structures. While itself not formally verified, the \gls{ebpf} verifier
facilitates the adoption of new \gls{ebpf} programs into production use cases, since an \gls{ebpf}
program is far less likely to adversely impact a production system than other methods of
extending the kernel (e.g.\ kernel patches and loadable kernel modules). In fact, \gls{ebpf} is
already being used in production at large datacenters by Facebook, Netflix, Google, and
others to monitor server workloads for security and performance
regressions~\cite{gregg2019_bpf}. These factors make \gls{ebpf} a promising choice for
designing an \textit{adoptable} security mechanism.

In summary, \gls{ebpf} offers unique and promising advantages for developing novel security
mechanisms. Its lightweight execution model coupled with the flexibility to monitor and
aggregate events across userspace and kernelspace provide the ability to control and audit
nearly any aspect of the running system. \gls{ebpf} maps, shareable across programs and between
userspace and the kernel offer a means of aggregating data from multiple sources at
runtime and using it to inform policy decisions across domains. A security mechanism based
on \gls{ebpf} can be dynamically loaded into the kernel as needed, and \gls{ebpf}'s safety guarantees
combined with its increasing adoption in production use cases provide strong adoptability
advantages. This means that a security mechanism based on \gls{ebpf} can be both adoptable and
effective.


\section{Contributions}%
\label{s:contributions}

This thesis offers several contributions to the fields of computer science, computer security, and
confinement. These contributions are as follows.
\begin{itemize}
  \item We present a novel framing of the confinement problem
  (\Cref{c:confinement-problem}) in the context of Linux, arguing that inherent complexity
  and misuse of existing primitives has led to semantic gaps in confinement. We argue that
  these gaps, in turn, impact security by encouraging the adoption of overly-generic
  polices that impact each other in unforeseen ways.

  \item We present the design and implementation of two \gls{ebpf}-based confinement
  engines, \bpfbox{} (\Cref{c:bpfbox}) and \bpfcontain{} (\Cref{c:bpfcontain}). The former
  is a prototype for \gls{ebpf}-based confinement and the latter extends \bpfbox{},
  improving its security and introducing a model for container-specific confinement.
  \bpfbox{} and \bpfcontain{} are the first high-level, \gls{ebpf}-based confinement
  frameworks of their kind.

  \item We evaluate (\Cref{c:evaluation}) \bpfbox{} and \bpfcontain{} in the context of
  their performance overhead and security. Specifically, we present results from
  benchmarks along with an informal security analysis. We also discuss how extensions on
  top of \bpfbox{} and \bpfcontain{} could improve their performance and security in the
  future.

  % \item We discuss how \bpfbox{} and \bpfcontain{} can be used to implement several types
  % of confinement policy (\Cref{c:case-studies}), providing policy examples along with
  % detailed comparisons therein.

  % \item We examine (\Cref{c:discussion}) how future iterations on \bpfcontain{} can
  % further improve its confinement guarantees, performance overhead, and general utility.
  % We also discuss how both \bpfbox{} and \bpfcontain{} are useful as proofs of concept,
  % and how \gls{ebpf} might be used to implement other security mechanisms in the future.
\end{itemize}

\section{Outline}%
\label{s:outline}

The rest of this thesis proceeds as follows. \Cref{c:background} presents detailed
background information on virtualization, confinement, operating system security, and
\gls{ebpf}.  \Cref{c:confinement-problem} presents a novel framing of the confinement
problem, outlining the motivation and design goals for \bpfbox{} and \bpfcontain{}, and
presenting a threat model for confinement.  \Cref{c:bpfbox} describes the design and
implementation of the initial \bpfbox{} prototype and documents its original policy
language. \Cref{c:bpfcontain} describes the design and implementation of \bpfcontain{},
discusses how it has evolved from \bpfbox{}, and highlights opportunities for future
extensions that can make \bpfcontain{} more useful for confining containers.

\Cref{c:evaluation} presents an evaluation of the \bpfbox{} and \bpfcontain{} prototypes
from the perspective of performance and security. We present benchmarking data comparing
both systems with AppArmor, a popular \glsentryfull{lsm} implementation of mandatory
access control. We also present a security analysis of \bpfbox{} and \bpfcontain{},
highlighting areas of weakness and specific aspects of \bpfbox{} upon which \bpfcontain{}
improves. \Cref{c:case-studies}  presents policy examples for \bpfbox{} and \bpfcontain{}
and provides a detailed comparison, highlighting the strengths and weaknesses of each.
\Cref{c:discussion} concludes with a high-level discussion on \bpfbox{} and \bpfcontain{},
including limitations and opportunities for future work.

% \begin{inprogress}
% The rest of this thesis proceeds as follows. \Cref{c:background} presents background on
% virtualization and process confinement, historical confinement techniques that have been
% employed in Linux and other Unix-like operating systems, and \gls{ebpf} and its applications to
% performance and security monitoring. \Cref{c:bpfbox} describes the design and
% implementation of \bpfbox, presents an initial performance evaluation, and~\todo{What
% else?  Need to figure out how much I want to say about \bpfbox}. \Cref{c:bpfcontain}
% presents \bpfcontain, an iteration on the original \bpfbox{} system designed
% specifically for container security. We present its design and implementation, evaluate
% its performance and effectiveness as a security mechanism in greater detail, and describe
% how it can be applied to secure practical container deployments. \Cref{c:discussion}
% presents a discussion on \bpfbox{} and \bpfcontain{} and discusses opportunities for
% future work both in the context of container security and other potential security
% applications of \gls{ebpf} beyond the scope of confinement. \Cref{c:related} surveys existing
% literature in the confinement space and compares \bpfbox{} and \bpfcontain{} to extant
% approaches. \Cref{c:conclusion} concludes.
% \end{inprogress}
