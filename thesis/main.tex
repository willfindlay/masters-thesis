\documentclass[
  fontsize=12pt,
  titlepage=firstiscover,
  paper=letter,
  %twoside,
  oneside,
  cleardoublepage=plain,
  parskip=half-,
  DIV=10,
  parindent,
  appendixprefix,
  chapterprefix,
  listof=totoc,
  %final
]{scrbook}

\usepackage[utf8]{inputenc}

\usepackage[acronym, toc, numberedsection=nolabel, section=chapter]{glossaries}
\setacronymstyle{short-long}
\makeglossaries
\newacronym{os}{OS}{Operating System}
\newacronym{bpf}{BPF}{Berkeley Packet Filter}
\newacronym{ebpf}{eBPF}{Extended \glsentryshort{bpf}}
\newacronym{cbpf}{cBPF}{Classic \glsentryshort{bpf}}
\newacronym{btf}{BTF}{\glsentryshort{bpf} Type Format}
\newacronym{jit}{JIT}{Just-In-Time}
\newacronym{core}{CO-RE}{Compile Once, Run Everywhere}
\newacronym{dac}{DAC}{Discretionary Access Control}
\newacronym{acl}{ACL}{Access Control List}
\newacronym{mac}{MAC}{Mandatory Access Control}
\newacronym{mls}{MLS}{Multi-Level Security}
\newacronym{lsm}{LSM}{Linux Security Modules}
\newacronym{krsi}{KRSI}{Kernel Runtime Security Instrumentation}
\newacronym{usdt}{USDT}{User Statically Defined Tracepoints}
\newacronym{pid}{PID}{Process ID}
\newacronym{tid}{TID}{Thread ID}
\newacronym{tgid}{TGID}{Task Group ID}
\newacronym{uts}{UTS}{Unix Timesharing System}
\newacronym{uid}{UID}{User ID}
\newacronym{gid}{GID}{Group ID}
\newacronym{euid}{EUID}{Effective \glsentrylong{uid}}
\newacronym{egid}{EGID}{Effective \glsentrylong{gid}}
\newacronym{tcb}{TCB}{Trusted Computing Base}
\newacronym{cots}{COTS}{Commercial Off-The-Shelf}
\newacronym{ipc}{IPC}{Inter-Process Communication}
\newacronym{ip}{IP}{Internet Protocol}
\newacronym{toctou}{TOCTTOU}{Time of Check to Time of Use}
\newacronym{isa}{ISA}{Instruction Set Architecture}
\newacronym{fpga}{FPGA}{Field-Programmable Gate Array}
\newacronym{abi}{ABI}{Application Binary Interface}
\newacronym{api}{API}{Application Programming Interface}
\newacronym{aslr}{ASLR}{Address Space Layout Randomization}
\newacronym{kaslr}{KASLR}{Kernel \glsentryshort{aslr}}
\newacronym{mmu}{MMU}{Memory Management Unit}
\newacronym{tlb}{TLB}{Translation Lookaside Buffer}
\newacronym{vm}{VM}{Virtual Machine}
\newacronym{lxc}{LXC}{Linux Containers}
\newacronym{cpu}{CPU}{Central Processing Unit}
\newacronym{gui}{GUI}{Graphical User Interface}
\newacronym{oci}{OCI}{Open Container Initiative}
\newacronym{cgi}{CGI}{Common Gateway Interface}
\newacronym{vfs}{VFS}{Virtual Filesystem}

\usepackage{findlay}
\usepackage{langs}
\usepackage{listings-rust}
\usepackage{epigraph}
\usepackage{circledsteps}
\usepackage{subfig}

% for changing margins in titlepage
\RequirePackage{geometry}

% for 1.5 line spacing
\RequirePackage{setspace}
\doublespacing%
% single spacing for table of contents
\AfterTOCHead{\singlespacing}

% recompute page layout based on the above
\recalctypearea%

% Use rmfamily for chapters, sections, etc.
\addtokomafont{disposition}{\rmfamily}

% Fix top and bottom margins
%\newgeometry{margin=1in}

% Uncomment this to help debug layout
%\usepackage{showframe}

\setcapindent{0pt}

\newcommand{\bpfbox}{\textsc{BPFBox}}
\newcommand{\bpfcontain}{\textsc{BPFContain}}

\addbibresource{refs.bib}

\title{A Practical, Lightweight, and Flexible Confinement Framework in eBPF}
\author{William P.\ Findlay}
\date{August, 2021}

\hyphenation{App-Armor}

% Remove spurious dot in figure/table captions
\renewcommand*{\figureformat}{%
  \figurename~\thefigure%
}
\renewcommand*{\tableformat}{%
  \tablename~\thetable%
}

\renewcommand{\lstlistlistingname}{List of Code Listings}

% Custom labels for research questions
\newlist{rqenum}{enumerate}{1}
\setlist[rqenum]{label=\textbf{RQ\arabic*}, ref=\textbf{RQ\arabic*}}
\crefname{research question}{Research Question}{Research Questions}
\crefalias{rqenumi}{research question}

\begin{document}

% ==============================================================================
% Cover Page, Abstract and TOC
% ==============================================================================

% Title page
\makeatletter
\begin{titlepage}
    \newgeometry{margin=0.8in}
    \begin{center}
      \vspace*{1cm}
      {\LARGE\bfseries \@title}

      \vspace{1cm}
      by
      \vspace{1cm}

      {\itshape\large \@author\/}

      \vfill

      A thesis submitted to the Faculty of Graduate and Postdoctoral Affairs\\
      in partial fulfillment of the requirements for the degree of

      \vspace{3cm}
      {\bfseries Master of Computer Science}
      \vspace{3cm}


      {\@date}
      \vspace{0.5cm}

      Carleton University\\
      Ottawa, Ontario
      \vspace{0.5cm}

      \copyright{}~2021 \@author%
    \end{center}
\end{titlepage}
\makeatother
\cleardoublepage

\renewcommand*{\titlepagestyle}{plain}

\frontmatter%

\vspace*{6em}
\begin{center}
\textit{To my parents and grandparents, for believing in me\\even when I didn't believe in myself.}
\end{center}
\vspace*{\fill}

\chapter*{Abstract}%
\addcontentsline{toc}{chapter}{Abstract}%
\begingroup
\small
Confining operating system processes is essential for preserving least-privilege access to
system resources, hardening the system against successful exploitation by malicious
actors. Classically, confinement on Linux has been accomplished through a variety of
disparate confinement primitives, each targeting a different aspect of process behaviour
and each with its own set of policy semantics. This has led to difficulties in realizing
practical confinement goals, due to the complexities, inter-dependence relationships, and
semantic gaps that arise from recombining existing confinement primitives in unintended
ways. Linux containers are a particularly poignant example of this phenomenon, with
existing container security policies often being overly complex and overly permissive in
practice.

To better isolate user processes and achieve practical confinement goals, we argue that
novel confinement mechanisms are needed to bridge the semantic gap between security policy
and enforcement. We hypothesize that a new Linux kernel technology, eBPF, enables the
creation of precisely such a confinement mechanism. eBPF programs can be dynamically
loaded into the kernel by a privileged process and are checked for safety before they run
in kernelspace.  This approach affords an opportunity to create an adoptable,
container-specific confinement mechanism without tying the kernel down to one specific
implementation.  Further, an eBPF-based confinement solution can be loaded and unloaded at
runtime, without updating or even restarting the operating system kernel; this property
enables rapid prototyping and debugging, similar in spirit to how we debug userspace
applications in practice.

In this thesis, we present the design and implementation of two novel confinement
solutions based on eBPF, \bpfbox{} and its successor, \bpfcontain{}. We discuss issues in
the Linux confinement space that motivated the creation of \bpfbox{} and \bpfcontain{},
discuss policy examples, and present the results of a performance evaluation and informal
security analysis. Results from this research indicate that \bpfbox{} and \bpfcontain{}
incur modest overhead despite their increased flexibility over existing Linux security
solutions.  We also find that there is significant opportunity to improve \bpfbox{} and
\bpfcontain{}, and to introduce future security mechanisms based on eBPF\@.

\endgroup
\cleardoublepage%

% \chapter*{Acknowledgements}%
% \addcontentsline{toc}{chapter}{Acknowledgements}%
% \begingroup
% \small
% \enlargethispage{2\baselineskip}
% I would first and foremost like to thank my thesis supervisors, Dr.\ Anil Somayaji and
% Dr.\ David Barrera for their constant support, sage advice, and invaluable feedback
% (particularly on early drafts of this document). I feel confident in saying that I would
% not have reached this point in my graduate school career if not for their dedication and
% encouragement. I am also grateful to my other committee members, Dr.\ Lianying (Viau) Zhao
% and Dr.\ Paula Branco for taking the time to read and evaluate my work.

% I would also like to thank the professors and fellow members of the CCSL/CISL sister labs
% for their valuable feedback on early iterations of my work, and for providing
% a stimulating environment to learn, grow, and foster my passion for operating system security.

% I am indebted to the many valuable members of the BPF and Linux kernel development
% community, whose hard work and dedication to free and open source software is reflected in
% the very foundations of this research. In particular, I would like to acknowledge Alexei
% Starovoitov and Daniel Borkmann for creating eBPF, Andrii Nakryiko for his work on
% \textit{libbpf} and CO-RE, and K.P.\ Singh for his work on bringing LSM hook support to
% BPF\@. Many other members of the BPF community have proved invaluable sources of
% inspiration and guidance throughout my academic career. While they are too many to name
% here, I appreciate them all the same.

% Lastly, I would like to thank my friends and family for their continued and unwavering
% support throughout this endeavour (and for many more endeavours to come).
% \endgroup
% \cleardoublepage%

\chapter*{Prior Publication}%
\addcontentsline{toc}{chapter}{Prior Publication}%
\begingroup
\small
A publication and pre-print have arisen as a direct result of the research in this thesis.
While these works represent joint contributions of all authors, any sections reproduced in
this thesis represent the sole work of the thesis author, with editorial and positioning
contributions by co-authors. Each work is listed below.

\Cref{c:bpfbox} contains text and ideas from our paper \enquote{\bpfbox{}: Simple Precise
Process Confinement in \gls{ebpf}}~\cite{findlay2020_bpfbox}, co-authored with Anil
Somayaji and David Barrera, and published at the Cloud Computer Security Workshop (CCSW)
2020 as part of the ACM CCS conference.

\Cref{c:bpfcontain} contains some text and ideas from our paper \enquote{\bpfcontain{}:
Fixing the Soft Underbelly of Container Security}~\cite{findlay2021_bpfcontain},
co-authored with David Barrera and Anil Somayaji. An early draft of this work is available
on the ArXiV pre-print database, although it differs substantially from the version
presented in this thesis.
\endgroup
\cleardoublepage%

% TOC
\begingroup
\hypersetup{linkcolor=black}
\tableofcontents
\begin{singlespace}
\listoffigures
\listoftables
\lstlistoflistings
\end{singlespace}
\endgroup

\mainmatter%

% ==============================================================================
% Main Chapter Content Here
% ==============================================================================

\chapter{Introduction}%
\label{c:introduction}
Virtualization is not confinement. To security experts, this may be an obvious statement,
but these two concepts are often conflated, leading to dangerous assumptions about
security in practice. To see why virtualization and confinement are disparate concepts,
consider the goals of each. \textit{Virtualization} describes the goal of providing
a unique, private mapping of shared system resources to a particular subject~\todo{CITE}
(be it a process, a virtualized operating system, or something else).
\textit{Confinement}, on the other hand, describes the goal of restricting a subjects
access to system resources or other subjects~\todo{CITE}. In other words, virtualization
is about \textit{what we can see}, whereas confinement is about \textit{what we can do}.
It should now be abundantly clear that virtualization and confinement are not only two
entirely different concepts, but that virtualization must be combined with confinement to
offer any practical security guarantees.

\begin{inprogress}
  Container technologies on Linux offer a motivating example of the difference between
  virtualization and confinement and how conflating the two can result in problematic
  misunderstandings about the security of a system. Linux containers offer lightweight
  virtualization and confinement using a series primitives exposed by the operating system
  kernel. Namespaces and cgroups virtualize system resources while confinement layers such
  as \texttt{seccomp(2)} provide some degree of isolation from the rest of the system~\todo{CITE}.
  Unfortunately, while virtualization primitives are widely used in container deployments,
  the use of confinement primitives is often overlooked, oversimplified, or overly
  permissive (i.e.~misconfigured)~\todo{CITE}.
\end{inprogress}


\section{Contributions}%
\label{s:contributions}


\chapter{Background and Related Work}%
\label{c:background}
This chapter presents background information on virtualization and confinement
technologies in Linux and other Unix-like operating systems. It also presents a detailed
history and examination of \gls{ebpf} and discusses its applications in the domains of computer
security, performance monitoring, and beyond.

\section{Process-Level Virtualization and Confinement}%
\label{s:virtualization-and-confinement}

Unlike virtual machines, containers are nothing but a collection of related processes
(sometimes just one process, but often more) running on the host operating system. These
processes share the same kernel as other running containers and as \enquote{native}
processes running outside of a container. While these properties can help containers to be
lighter-weight than virtual machines, they also necessitate the usage of process-level
virtualization and confinement primitives to isolate containers from one
another~\cite{sultan2019_container_security}.  \Cref{s:container-security-bg} covers
containers and container security in greater detail, further differentiating them from
VMs. For the purposes of this section, it is enough to know that containerized processes
run directly on the host, share the host kernel, and use virtualization and confinement
primitives to achieve isolation.

While containers rely on virtualization and confinement primitives to facilitate normal
operation, this notion of process isolation predates the advent of containers by decades.
Isolation of processes, particularly in the case of multi-tenant and network-facing
systems, makes a lot of sense from the persecutive of security best-practices. Locking
down a program such that it can only perform the actions required for it to function
correctly can help reduce attack surface and minimize the potential for damage to the rest
of the system in the event of compromise. This notion of restricting a program's
capabilities to the minimal viable set has a name in the security literature: the
\textit{principle of least-privilege}~\cite{schneider03_least_privilege,
van_oorschot2020_tools_jewels}.

Researchers have been studying confinement for decades~\cite{lampson1973_confinement}, and
have been designing and applying confinement primitives since the early days of
time-sharing computers and multi-tenant systems~\cite{shu2016_security_isolation_study}.
Process-level virtualization has become similarly important, but for subtly different
reasons\,---\,namely, for the purpose of restricting and modifying a process' visibility
into the rest of the system.  Decades of research into virtualization and confinement
techniques has culminated in a wealth of knowledge and prior artifacts, some of which have
seen widespread deployment and remain relevant to date. The rest of this section presents
technical background on such process-level virtualization confinement mechanisms and
surveys the academic literature related to this area.

\subsection{Virtual Memory and Memory Protection}%
\label{ss:virtual-memory}

\subsubsection*{CPU Protection Rings}

\begin{inprogress}
  \begin{itemize}
    \item Multics protection rings
    \item Adoption of protection rings by modern CPUs (simplified model, rings 0-3)
    \item CPU supports rings lower than 0 (really just virtualizing ring 0) for the hypervisor, CPU engine
    \item Userspace, kernelspace
  \end{itemize}
\end{inprogress}

\subsubsection*{Virtual Memory}

\begin{inprogress}
  \begin{itemize}
    \item Page tables, page table walk
    \item CPU's MMU, maps virtual addresses to physical addresses
    \item Shared memory
    \item Memory protection bits
    \item ASLR, KASLR
  \end{itemize}
\end{inprogress}

\subsection{The Reference Monitor}%
\label{ss:refmon}

The \textit{reference monitor concept}, first introduced in the landmark 1972 Anderson
Report~\cite{anderson1972_report}, was among the earliest complete descriptions of a full
access control mechanism and remains influential in operating system design to date. The
reference monitor is an abstract model for a secure reference validation mechanism built
into the operating system. The model partitions the system into \textit{subjects} (users,
processes, etc.) and \textit{objects} (system resources).  Subjects request access to
objects and the reference monitor checks this access against a known list of allowed
accesses, parameterized by the subject, object, and requested access. The software
implementation of a reference monitor is known as the \textit{security kernel.}
\Cref{fig:refmon} depicts the reference monitor concept as it was first presented by
Anderson~\cite{anderson1972_report}.

\begin{figure}[tbp]
  \centering
  \includegraphics[width=0.6\linewidth]{figs/background/refmon.pdf}
  \caption[The reference monitor concept]{
    The reference monitor concept as outlined in the Anderson
    Report~\cite{anderson1972_report}. User processes make requests (e.g.~via system calls
    to the operating system). The \gls{os} kernel invokes the reference monitory, which is
    implemented in software as a security kernel. The reference monitor queries its
    security policy taking the subject, object, and other parameters as input. As output,
    it returns a security decision (i.e.~whether the requested access should be
    \textit{allowed} or \textit{denied}).
  }%
  \label{fig:refmon}
\end{figure}

While the majority of modern operating systems do not include a security kernel as
described by Anderson, the reference monitor architecture has informed the design of
modern access control mechanisms and models the reference validation process that occurs
when the kernel is servicing userspace requests (i.e.~system
calls)~\cite{van_oorschot2020_tools_jewels}. In order for such a design to be considered
valid, Anderson enumerate three key properties: (i) Tamper Resistance; (ii) Complete
Mediation; and (iii) Verifiability. These properties can inform the way we reason about
and design modern access control mechanisms, even if they do not strictly adhere to the
reference monitor model.

\paragraph*{Tamper-Resistance}

In order for the reference monitor to be considered \textit{tamper-resistant}, an unauthorized
party must not be able to alter the reference monitor's code or modify any data
(e.g.~memory, persistent storage) that the reference monitor relies on to enforce correct
reference validation~\cite{anderson1972_report}. This property follows from the fact that
unauthorized tampering with the reference monitor totally invalidates any security
guarantees.

\paragraph*{Complete Mediation}

The property of \textit{complete mediation} means that the reference monitor should be
invoked on all security sensitive events. It should be impossible for an attacker to
bypass the reference monitor in any way. Any software that is not subject to reference
validation should be considered a part of the reference
monitor~\cite{anderson1972_report}.

\paragraph*{Verifiability}

\textit{Verifiability} refers to the ability to reason about or prove the correctness of
the reference monitor (i.e. that the first two properties hold). Formal verification methods
are the best way of achieving verifiability, although this may not necessarily be practical
for highly complex systems. For this reason, it is recommended to design the reference monitor
in such a way that verifiability is maximized~\cite{anderson1972_report}.

\subsection{Discretionary Access Control}%
\label{ss:dac}

\textit{Discretionary access control} (DAC) forms the most basic form of access control in
many operating systems, including Linux and other Unix-like operating systems, and
Microsoft Windows. First formalized in the 1983 US Department of Defense (sic)
standard~\cite{orange_book}, a discretionary access control mechanism partitions and
labels system objects (i.e.~resources such as files) by the subjects (i.e.~actors such as
users and user processes) that \textit{own} them. The corresponding resource owner then
has full authority to decide which subjects have access to its owned objects. This notion
of ultimate authority over a subject's owned objects constitutes the primary difference
between discretionary access control and mandatory access control, which is covered in
\Cref{ss:mac}.

Classically, Unix-like systems have implemented discretionary access control in the form
of \textit{permission bits} and \textit{access control lists}. Each process on the system
runs under a specific user and group ID, which uniquely identify the user and group of the
process respectively, where each group is a collection of one or more users. Permission
bits and access control lists denote access permissions according to the user ID and group
ID of the process requesting the resource. These permissions can in turn be overridden by
the \textit{superuser} or \textit{root}~\cite{van_oorschot2020_tools_jewels,
jaeger2008_os_security}.

\subsubsection*{Permission Bits}

Permission bits in Unix are special metadata associated with a file that determine
coarse-grained access to the file according to a subject's \gls{uid} and \gls{gid}.
Permission bits are divided into three sections: \textit{User}, \textit{Group}, and
\textit{Other}. The \textit{User} bits apply to subjects whose \gls{uid} matches the
resource owner's \gls{uid}, while the \textit{Group} bits consider the \gls{gid} instead.
In all other cases (i.e.~when neither the \gls{uid} nor the \gls{gid} matches), the
\textit{Other} bits determine the allowed access. To determine which access should be
allowed, permission bits encode a coarse-grained \textit{access vector}, specifying read,
write, and execute access on a file or directory (in the case of a directory, execute
access implies the ability to \texttt{chdir(2)} into that directory).

While convenient, permission bits are generally insufficient to provide legitimate
security guarantees to modern systems~\cite{van_oorschot2020_tools_jewels,
jaeger2008_os_security}. In particular, permission bits encode coarse-grained permissions
and apply these permissions in a coarse-grained, all-or-nothing, manner. For instance,
consider the use case of granting read-only access to another user. Specifying such access
as part of the \textit{Other} bitmask implies granting access to any user on the system.
Specifying access to a particular \textit{Group} is slightly better, but the resource
owner has no direct control over which other users belong to this group, now or in the
future. Thus, we cannot say with certainty that we may specify such access without
violating our security assumptions.

\subsubsection*{Access Control Lists}

\Glspl{acl} offer a slightly more granular alternative to permission bits,
at the expense of increased complexity~\cite{jaeger2008_os_security,
van_oorschot2020_tools_jewels}. Unlike permission bits, which rely on three coarse-grained
subject categories (\textit{User}, \textit{Group}, and \textit{Other}), an access control
list defines a set of subjects and their corresponding permissions for every object. It
may be helpful to think of this as breaking up the \textit{Other} category into distinct
subjects rather than granting or revoking blanket access to all other users on the system.

Capability lists, complementary to access control lists, define a set of objects and
allowed access patterns for every subject. A capability list for a given subject can be
derived by taking the set of all access control lists for every object and vice
versa~\cite{van_oorschot2020_tools_jewels}. Together, the set of all access control lists
(or capability lists) forms an \textit{access matrix}, describing the \gls{dac} policy over
the entire system. \Cref{fig:acl} depicts this relationship.

\begin{figure}[tbp]
  \centering
  \includegraphics[width=0.8\linewidth]{figs/background/acl.pdf}
  \caption[The access matrix]{
    The access matrix and the relationship between \glspl{acl} and capability
    lists~\cite{anderson1972_report, van_oorschot2020_tools_jewels, jaeger2008_os_security}.
  }%
  \label{fig:acl}
\end{figure}

\subsubsection*{The Superuser and Setuid}

To facilitate system administration, many \gls{dac} schemes incorporate the notion of
a \textit{superuser} or \textit{administrator role} into their model. In Unix and
Unix-like operating systems, the superuser or \textit{root} user is denoted by the
\gls{uid} of zero. Any process running with the \gls{euid} of zero is said to be
\textit{root-privileged}. These root-privileged processes can then override the system's
\gls{dac} policy, bypassing permission bits and access control entries on system objects.

In many cases, a program requires additional privileges in order to function. For
instance, a \texttt{login} program would require the ability to read security-sensitive
password entries in \texttt{/etc/shadow}. To achieve such functionality, Unix provides
special \texttt{setuid} and \texttt{setgid} permission bits that implicitly set the
effective user and group IDs of a process to those of the file owner.
A sufficiently-privileged process may also change its own \gls{euid} or \gls{egid} at
runtime using the \texttt{setuid(2)} and \texttt{setgid(2)} family of system calls. Our
login program, for instance, could use these system calls to drop its privileges to those
of the user being logged in. While necessary under the Unix \gls{dac} model, setuid and
setgid binaries have long been the target of exploitation, particularly for privilege
escalation attacks~\cite{dittmer2014_setuid, van_oorschot2020_tools_jewels,
jaeger2008_os_security}.

\subsubsection*{User and Group Assignment}

To alleviate concerns with discretionary access control, systems often take the approach
of assigning a unique user and/or group to a specific application. Such applications are
typically security-sensitive, such as a privileged daemon or network-facing service. This
technique achieves a dual-purpose: firstly, the application can lock down any resources it
owns, simply by restricting any access to its own \gls{uid}; secondly, the resulting
process no longer needs to run under the same \gls{uid} as its parent. This effectively
limits the amount of outside resources that the application can access (so long as
permission bits are correctly configured). In a sense, such a model approaches role-based
access control~\todo{CITE} (covered in \Cref{ss:rbac}).

The Android operating system takes this model a step further, assigning a unique \gls{uid}
and \gls{gid} to every application on the system, with optional \gls{uid} sharing between
applications that come from the same vendor. Under this model, no process' \gls{uid} ever
corresponds to a human user. While this arguably improves security, Barrera
\etal~\cite{barrera2012_android} found weaknesses in Android's \gls{uid} sharing model
that can reduce its security to the trustworthiness of an app's signing key.

\subsubsection*{POSIX Capabilities}

POSIX capabilities~\cite{posix_capabilities, corbet2006_capabities_a,
corbet2006_capabities_b} are highly related to Unix DAC in the sense that they were
originally designed to break up the multitude of privileges associated with the
\textit{root} user into more manageable pieces. In this sense, POSIX capabilities (when
properly used) are more conducive to the principle of least-privilege. A process need not
necessarily possess full root-level access to the system when only a small subset of those
privileges are actually required.

Originally specified in the (now withdrawn) 1003.1e POSIX standard, POSIX capabilities
were only ever (partially) implemented on Linux~\cite{anderson2017_comparison}. Other
Unix-like operating systems prefer alternative methods of restricting privileges, many of
which are discussed in \Cref{ss:syscall-filtering}. POSIX capabilities specify three
\textit{capability sets} for a given process: the \textbf{bounding set}, the
\textbf{inheritable set}, and the \textbf{effective set}. The bounding set determines the
set of all capabilities that a process is ever allowed to possess. The inheritable set
determines the set of all capabilities that can be inherited across \texttt{execve} calls.
Finally, the effective set determines the set of capabilities that a process can use
(i.e.~which capabilities a process currently possesses).

Linux exposes POSIX capabilities through extended filesystem attributes, much the same way
that \glspl{acl} are implemented~\cite{corbet2006_capabities_b}. These file-based
capabilities function in a similar manner to the setuid bit, implicitly setting the
bounding, inheritable, and effective capability sets on execution. In addition so
supporting capabilities as extended filesystem attributes, the kernel also supports
dropping specific capabilities from each of the three sets through the \texttt{ptrctl(2)}
system call. This enables a higher-privileged process (e.g.~running as root) to drop
elevated privileges while retaining those it needs to function. As of Linux 5.12, the
kernel supports 41 capabilities in total, including the all-encompassing
\texttt{CAP\_SYS\_ADMIN}~\cite{linux_capability_h}.

It is worth mentioning that the term \enquote{POSIX capabilities} does \textit{not}
describe capabilities as they are broadly defined by operating system security
researchers~\cite{anderson2017_comparison}. In particular, Dennis and Van
Horn~\cite{dennis1966_semantics} first defined the notion of capabilities as a means of
restricting access to pointers, guarding references to system objects. Unlike the
capabilities defined by Dennis and Van Horn, POSIX capabilities are not associated with
any given system object. Dennis and Van Horn's capabilities more closely resemble that of
the access matrix introduced by Anderson~\cite{anderson1972_report} and similar mechanisms
have been implemented in other systems such as FreeBSD's
Capsicum~\cite{watson2010_capsicum} and the CHERI architecture~\cite{watson2015_cheri,
davis2019_cheriabi}. These are discussed in more detail in \Cref{ss:syscall-filtering}.

\subsubsection*{DAC Security Assumptions and Attacks}

Although discretionary access control provides a convenient and intuitive user-centric
model for object ownership and permissions, it makes some dangerous assumptions about
security that can totally invalidate the model in
practice~\cite{shu2016_security_isolation_study}. In particular, DAC assumes that all
processes are benign and contain no exploitable vulnerabilities. The mere existence off
malware and exploitable vulnerabilities (e.g.~memory safety vulnerabilities) immediately
invalidates this assumption. For instance, consider an honest but vulnerable piece of
software running under a given \gls{uid} $X$. An attacker exploiting a vulnerability in
this application could perform arbitrary operations on any files owned by $X$. Similarly,
a Trojan horse\footnote{A Trojan horse is a piece of ostensibly benign software that
is designed to perform some malicious action or actions in addition to its ordinary
functionality~\cite{van_oorschot2020_tools_jewels}.}~\cite{shu2016_security_isolation_study,
van_oorschot2020_tools_jewels} can perform arbitrary malicious operations on $X$'s files
without needing to exploit any vulnerability. The fundamental issue with Unix \gls{dac} is
that these files need not necessarily have \textit{anything} to do with the program in
question.

Another fundamental issue with Unix \gls{dac} lies in the ultimate authority of the root
user. Any process running with \gls{euid}=0 is immediately part of the system's
\textit{\gls{tcb}}\footnote{The \textit{trusted computing base} is the set of all hardware
and software that must be trusted in order for the system to be considered trusted.
Typically, this includes system hardware, the operating system itself, and a small subset
of userspace programs~\cite{jaeger2008_os_security}.}. The same applies to any executable
marked as setuid root. Processes that run with root privileges are prime targets for
attacker exploit, since a successful attack can effectively compromise the entire system.
For instance, confused deputy attacks~\cite{hardy1988_confused_deputy,
shu2016_security_isolation_study} can exploit privileged processes by tricking them into
performing some undesired action. The coarse granularity of Unix \gls{dac} renders it
particularly vulnerable against such attacks.

\subsubsection*{Proposals for Alternative Schemes}

Academics have long recognized that weaknesses in the discretionary access control model
must be addressed. Many have turned to mandatory access control~\todo{CITE ALL EXAMPLES}
(c.f.~\Cref{ss:mac}) to solve the fundamental issues in \gls{dac}, while others have
proposed improvements or alternative schemes for implementing discretionary access
control~\todo{CITE ALL EXAMPLES}. This subsection focuses specifically on the latter.

Mao \etal~\cite{mao2009_trojan_resistant_dac} proposed IFEDAC as an alternative \gls{dac}
model that is resistant to Trojan horse attacks. The insight behind their work was that
\gls{dac}'s primary weaknesses lie in the inability to distinguish requests involving
multiple actors. Their mechanism proposes to track information flows between subjects and
use these flows to infer a list of subjects that have influenced a request.

Under the traditional Unix \gls{dac} model, only the \gls{uid} and \gls{gid} of the
process are considered when making access control decisions; under IFEDAC, the \gls{uid}
and \gls{gid} of the owner of the underlying executable would also be considered, along
with any other parties that may have influenced the state of the running process. This
approach is similar in spirit to taint tracking mechanisms~\todo{CITE}~\todo{FORWARD
REFERENCE}. To enable programs to function correctly, IFEDAC enables the user to define
\textit{exception policy} that specifies exceptions to IFEDAC enforcement. Mao
\etal~recommend that application authors and OS vendors should be responsible for
distributing such policies~\cite{mao2009_trojan_resistant_dac}.

Dranger, Solworth, and Sloan~\cite{solworth2004_layered_dac, dranger2006_dac_complexity}
presented a three-layered model of \gls{dac} mechanisms. The \textit{base layer} defines
the general access control model, while the \textit{parameterization layer} parameterizes
it according to deployment needs.  Finally, the \textit{local initialization layer}
comprises the set of subjects and objects along with their associated protections. The
authors showed that their model was generalizable and that it could be used to implement
any \gls{dac} mechanism.

Dittmer and Tripunitara~\cite{dittmer2014_setuid} examined the implementation and common
usage patterns of the POSIX setuid and setgid API across multiple Unix-like operating
systems. They identified weaknesses in systems that do not implement the latest POSIX
standard revisions and suggested that mismatched semantics between various implementors
can be a source of developer error. Finally, they presented an alternative API that
partitions \gls{uid} changes into permanent and temporary categories. Tsafrir
\etal~\cite{tsafrir2008_setuid} and Chen \etal~\cite{chen2002_setuid} identified the same
fundamental issues and proposed the adoption of similar mechanisms.


\subsection{Role-Based Access Control}%
\label{ss:rbac}

\begin{inprogress}
  \begin{itemize}
    \item
  \end{itemize}
\end{inprogress}



\subsection{Mandatory Access Control}%
\label{ss:mac}

In contrast with \gls{dac}, \textit{\gls{mac}} does not delegate permission assignment to
the resource owner~\cite{spencer1999_flask, van_oorschot2020_tools_jewels,
jaeger2008_os_security}. In the context of Unix, this means that \gls{mac} both overrides
traditional discretionary access controls \textit{and} applies access controls even to the
root user. Historical implementations of \gls{mac} have focused primarily on \gls{mls}, an
access control scheme that revolves around the \textit{secrecy} of objects and
\textit{access level} of subjects. In \gls{mls}, a subject may access an object with
a secrecy level less than or equal to its access level.
Multics~\cite{vyssotsky1965_multics, corbato1965_multics} was the first operating system
to pioneer the use of an \gls{mls} access control scheme.

While \gls{mls} is primarily applicable to military contexts, \gls{mac} has since evolved
into mainstream use through the advent of alternative implementations. The Flask architecture
introduced \todo{A PRACTICAL NOTION OF HOW MLS COULD BE APPLIED TO CONSUMER CONTEXTS?}

\subsubsection*{The Flask Architecture}

\begin{inprogress}
  \begin{itemize}
    \item Flask~\cite{spencer1999_flask}
  \end{itemize}
\end{inprogress}

\subsubsection*{Linux Security Modules}

\begin{inprogress}
  \begin{itemize}
    \item SELinux~\cite{smalley2001_selinux}, the reference
          policy~\cite{pebenito2006_refpol}
    \item SELinux policy generation efforts: Madison~\cite{macmillan07_madison},
          audit2allow~\cite{audit2allow}, guided policy generation~\cite{sniffen06_guided}
    \item AppArmor~\cite{cowan2000_apparmor}, aa-logprof, aa-genprof, and aa-easyprof
    \item Tomoyo
    \item FBAC-LSM
    \item FSF~\cite{hu2013_fsf}
    \item Landlock
    \item KRSI
  \end{itemize}
\end{inprogress}



\subsection{System Call Filtering and Capabilities}%
\label{ss:syscall-filtering}

\subsubsection*{Janus}
\label{sss:janus}

\subsubsection*{OpenBSD Pledge and Unveil}
\label{sss:pledge}

\subsubsection*{Linux Seccomp and Seccomp-BPF}%
\label{sss:seccomp}

\subsubsection*{FreeBSD Capsicum}
\label{sss:capsicum}

\begin{inprogress}
  \begin{itemize}
    \item Capsicum paper by Watson \etal~\cite{watson2010_capsicum}
  \end{itemize}
\end{inprogress}

\subsubsection*{CHERI Capabilities}
\label{sss:cheri}

\begin{inprogress}
  \begin{itemize}
    \item Original Cheri paper by Watson \etal~\cite{watson2015_cheri}
    \item Cheri ABI implementation in FreeBSD by Davis and Watson \etal~\cite{davis2019_cheriabi}
  \end{itemize}
\end{inprogress}



\subsection{Process-Level Virtualization}%
\label{ss:virtualization}

\subsubsection*{Chroots and Chroot Jails}

\subsubsection*{FreeBSD Jails}

\subsubsection*{Linux Namespaces}

\subsubsection*{Linux Cgroups}



%\subsection{Linux Security Modules}%
%
%\subsection{Process-Level Virtualization in Linux}%
%\label{ss:virtualization-bg}
%
%\subsection*{Namespaces}
%
%\subsection*{Process Control Groups}
%
%\subsection{Process Control Groups}%
%\label{ss:cgroups-bg}
%
%\subsection{Unix DAC}%
%\label{ss:unix-dac-bg}
%
%\subsection{Chroot Jails}%
%\label{ss:chroot-jails-bg}
%
%\subsection{FreeBSD Jails}%
%\label{ss:freebsd-jails-bg}
%
%\subsection{OpenBSD Pledge and Unveil}%
%\label{ss:pledge-bg}
%
%\subsection{Linux Seccomp and Seccomp-BPF}%
%\label{ss:seccomp-bg}
%
%\subsection{Linux MAC}%
%\label{ss:linux-mac-bg}







\section{Containers and Container Security}%
\label{s:container-security-bg}

\subsection{Containers}%
\label{ss:containers-bg}

\begin{figure}[tbp]
  \centering
  \includegraphics[width=0.8\linewidth]{figs/background/virtualization.pdf}
  \caption[A comparison of virtual machine and container architectures]{
    A comparison of virtual machine and container architectures. Containers \textbf{(a)}
    achieve virtualization using a thin layer provided by the host \gls{os}
    itself. They share the underlying operating system kernel and resources, requiring no
    guest \gls{os}. Type I hypervisors \textbf{(b)} virtualize and control the
    underlying hardware directly, but require full guest operating systems on top of the
    virtualization layer. Type II hypervisors \textbf{(c)} run on top of a host operating
    system but still require full guest operating systems above the virtualization layer.
  }%
  \label{fig:virt}
\end{figure}

\subsection{Container Security}%
\label{ss:container-security-bg}







\section{Extended BPF}%
\label{s:ebpf-bg}

\gls{ebpf} stands for \enquote{Extended \gls{bpf}}, though in reality it has very
little to do with Berkeley, packets, or filtering in its current
form~\cite{gregg2019_bpf}. In a nutshell, \gls{ebpf} is a Linux kernel technology that supports
dynamic system monitoring through the attachment of special \enquote{hooks} called \gls{bpf}
programs to specific kernel interfaces and userspace functions. In recent years, \gls{ebpf}'s
role has expanded, providing an interface to make extensions to the kernel as well as the
classic monitoring use case. In this section, we discuss the origins of \gls{ebpf}, its
components and how they work, its applications under the Linux kernel, and how it has
evolved over time.

\todo{Perhaps add a subsection here to discuss DTrace as a precursor to eBPF?}

\subsection{Origins of BPF\@: Efficient Packet Filtering and Beyond}%
\label{ss:origins-of-bpf-bg}

The original Berkeley Packet Filter, hereafter referred to as \gls{cbpf}\footnote{%
Throughout the rest of this thesis, we refer to extended \gls{bpf} using the terms
\enquote{\gls{ebpf}} and \enquote{\gls{bpf}} interchangeably. This is a matter of established
convention within the \gls{ebpf} community. Classic \gls{bpf} will be explicitly referred to by its
full name or the \gls{cbpf} acronym.}, arose out of a need to implement a more efficient packet
filtering mechanism for BSD Unix.  McCanne and Jacobson~\cite{mccanne1993_bpf} published
their work on \gls{cbpf} in 1993, marking an improvement over existing mechanisms in a number of
ways. Many of the reasons why classic \gls{bpf} was such an improvement over the status quo are
still relevant when discussing \textit{\gls{ebpf}}, and so we will briefly cover them here as
well.

In essence, classic \gls{bpf} is a \textit{register virtual machine} designed to take packets as
input and produce \textit{filtering decisions} as output. These filtering decisions could
then used to make decisions about whether a packet should be passed down to a more complex
pipeline for further analysis. The key insight behind \gls{cbpf} is that these filtering
decisions could be made more efficiently in \textit{kernelspace}, the part of the
operating system that runs in protection ring 0\footnote{Code that runs in ring 0 is said
to run with \textit{supervisor privileges} and is able to access all system memory. Ring
0 is the highest level of memory protection provided by the CPU~\cite{jaeger2008_os_security}.}
and which is most commonly associated with any parts of the operating system that do not
run in \textit{userland} (i.e.~the context of an ordinary user process). This provides
a considerable performance advantage over conventional approaches to network monitoring.
A typical network monitor runs in \textit{userspace}, meaning that packets need to be
copied over from kernelspace before they can be properly analyzed. This is an expensive
operation, requiring several context switches and potentially sleeping in the event of
a page fault~\cite{mccanne1993_bpf}.  By applying filtering logic in the kernel, this
expensive copying could be skipped for packets that would be discarded or ignored by the
network monitor anyway.

\begin{figure}[tbp]
  \centering
  \includegraphics[width=0.8\linewidth]{figs/background/classic-bpf.pdf}
  \caption[The classic BPF architecture]{The classic \gls{bpf} architecture. Adapted from McCanne and Jacobson~\cite{mccanne1993_bpf}.}%
  \label{fig:classic-bpf}
\end{figure}

Classic \gls{bpf} can be divided into two major components: a \textit{tap} mechanism and a set
of one or more \textit{filter} programs. The cBPF architecture is depicted in
\Cref{fig:classic-bpf}. \gls{cbpf} programs are expressed as a control-flow graph (CFG)
over a set of abstract registers, backed by physical registers on the CPU. The tap
mechanism hooks into packets as they enter the networking stack, copying and forwarding
them to the filters. At runtime, the filter programs walk their control-flow graph, taking
the forwarded packets as input. As output, they return a filtering decision which controls
whether or not the packet should be forwarded to userspace~\cite{mccanne1993_bpf}.

Since its original introduction in 1993, classic \gls{bpf} has since been ported to a number of
Unix-like operating systems, including Linux~\cite{linux_bpf}, OpenBSD~\cite{openbsd_bpf},
and FreeBSD~\cite{freebsd_bpf}. Classic \gls{bpf} forms the backbone of widely used traffic
monitoring tools, most notably tcpdump~\cite{tcpdump, mccanne1993_bpf}. In Linux, the
\texttt{seccomp(2)} system call~\todo{CITE Anderson} was enhanced to include classic \gls{bpf}
filters, allowing a user process to use classic \gls{bpf} programs to define allowlists and
denylists of system calls (c.f.~\Cref{sss:seccomp}).

In 2014, Alexei Starovoitov and Daniel Borkmann~\cite{starovoitov2014_ebpf} first proposed
a total overhaul of the Linux \gls{bpf} engine. Their proposal, dubbed \gls{ebpf}, expanded the
classic \gls{bpf} execution model into a full-fledged virtual instruction set. In particular,
the extensions included a 512 byte stack, 11 registers (10 of which are general-purpose),
the ability to call a set of allowlisted kernel helper functions, the ability to attach
programs to a variety of system events, specialized data structures (called \gls{bpf} maps) to
store and share data at runtime, and an in-kernel verification engine to check for program
safety. At runtime, programs can be dynamically attached to system events and are
just-in-time compiled into the native instruction set.  \Cref{fig:extended-bpf} depicts
the \gls{ebpf} architecture in detail. The reader is encouraged to compare this with the classic
BPF architecture, depicted in \Cref{fig:classic-bpf}.

\begin{figure}[tbp]
  \centering
  \includegraphics[width=0.8\linewidth]{figs/background/ebpf.pdf}
  \caption[The extended BPF architecture]{The extended \gls{bpf} architecture. Unlike classic
  \gls{bpf}, \gls{ebpf} programs are \gls{jit} compiled to the native instruction set, share data using
  specialized map data structures, and can be attached to many different kinds of system
  events. Programs can share data with each other and with the controlling userspace
  process using specialized map data structures. All \gls{ebpf} bytecode goes through
  a verification step before it can be loaded into the kernel.}%
  \label{fig:extended-bpf}
\end{figure}

While modern \gls{ebpf} has very little to do with the execution model of its older cousin, some
of the properties that made classic \gls{bpf} so performant still hold true today. In
particular, to notion of aggregating and processing data in kernelspace before
(optionally) handing it off to userspace is a key aspect of classic \gls{bpf} that has carried
over to \gls{ebpf}. What this means in practice is that \gls{ebpf} programs can be used to implement
very efficient monitoring software, harnessing the performance benefits of a pure
kernelspace implementation while maintaining the flexibility of a userspace
implementation.

\subsection{eBPF Programs}%
\label{ss:bpf-programs-bg}

\gls{ebpf} programs are expressed in a virtual RISC machine language called \gls{bpf} bytecode.  While
it is technically possible to write \gls{bpf} bytecode by hand, programs are most often compiled
from a restricted subset of the C programming language\footnote{Other languages may
eventually be used to write \gls{ebpf} programs as well.  For instance, an experimental \gls{ebpf}
target for the Rust programming language has recently been
proposed~\cite{decina2021_bpf_rust}. The important distinction here is that the set of all
possible \gls{ebpf} programs is a strict subset of the set of all possible programs.} using the
LLVM toolchain. Programs can be loaded and attached to system events using the
\texttt{bpf(2)} system call, at which point control passes to the \gls{ebpf} verifier, which
checks the programs to make sure they satisfy a set of safety
constraints~\cite{starovoitov2014_ebpf, gregg2019_bpf}. In particular, \gls{ebpf} programs must
consist of fewer than 1 million \gls{bpf} instructions and must not call into any kernel
functions outside of the allowlisted helpers. The program is also constrained to a 512
byte stack size; any additional memory required by the program must come from an \gls{ebpf} map
(c.f.~\Cref{ss:bpf-maps-bg}). For safety, memory accesses into allocated buffers must be
properly bounds checked, pointers must be null-checked before dereferencing, and any
access to external memory (e.g.~belonging to userspace programs or to the kernel itself)
must be read-only. Since \gls{ebpf} programs must provably terminate, no back-edges are
permitted in their control flow and all loops must be bounded by some fixed constant $i$
iterations.

To guard against data races, \gls{ebpf} programs always hold the kernel's RCU (read-copy-update)
lock while executing, gated by the \texttt{bpf\_prog\_enter} and \texttt{bpf\_prog\_exit}
functions in the kernel. In simple terms, the RCU lock allows concurrent reads, except in
the presence of updates, optimizing for read-mostly workloads (i.e.~precisely the sort of
workload \gls{ebpf} is designed for)~\cite{mckenney2007_rcu}. This implicitly enables \gls{bpf}
programs to read from many common kernel data structures without fear of data races and
simultaneously protects reads and updates to \gls{ebpf} maps, at a slight (albeit reasonable)
performance penalty~\cite{mckenney2007_rcu}. In addition to holding the RCU lock, \gls{ebpf}
programs are not considered \textit{preemptable} by default. In practice, this means that
\gls{ebpf} programs cannot sleep and must run to termination on their assigned core. This
property, while useful in many circumstances, enforces undesirable limitations on \gls{ebpf}
helpers, since it precludes any functionality that may cause the program to sleep (e.g.~a
page fault). To account for use cases where sleeping is unavoidable, Linux 5.10 introduced
sleepable versions of some \gls{ebpf} program types~\cite{starovoitov2020_sleepable}.

Once loaded into the kernel, \gls{ebpf} programs are represented as \gls{bpf} objects, each with its
own reference count. Loading a \gls{bpf} program and attaching it to a system event increments
the reference count, while detaching and unloading the program decrements the reference
count. The kernel also exposes a special filesystem, \gls{bpffs}, which allows \gls{bpf}
programs to be pinned. This also increments the reference count, allowing an attached
program to outlive its controlling process (i.e.~the process that loaded and attached
it)~\cite{gregg2019_bpf}.

\subsubsection*{Working with the Verifier}

In practice, the restrictions imposed by the verifier mean that \gls{ebpf} programs are not
\textit{Turing-complete}~\cite{gregg2019_bpf}.  This property is required, given that the
halting problem (i.e.~the decidability of program termination) is known not to be solvable
for Turing-complete programs. This notion of Turing-incompleteness means that the set of
all possible \gls{ebpf} programs is a strict subset of the set of all possible C programs. While
these limitations help to ensure program safety, they also naturally restrict some
operations which \textit{may} be safe but are not strictly verifiable. To overcome the
limitations imposed by the verifier and achieve this safe-yet-unverifiable behaviour, \gls{ebpf}
programmers have a few tools in their arsenal. For instance, a specific set of allowlisted
kernel helpers offers the ability to call into specific kernel functions, bypassing the
limitations imposed by the \gls{ebpf} verifier. As a simple example, the
\texttt{bpf\_probe\_write\_user()} helper allows an \gls{ebpf} program to write to a userspace
memory address, bypassing the read-only restrictions imposed by the verifier. While these
allowlisted helpers operate in a \textit{mostly} unrestricted context, their usage
\textit{is} restricted at the function call boundary, ensuring that the \gls{ebpf} program obeys
the safety contract specified by the helper function.  Another common design pattern is
using a dummy \gls{ebpf} map as a scratch buffer to reserve a larger amount of memory for the
\gls{ebpf} program.  Since \gls{ebpf} programs cannot sleep~\cite{gregg2019_bpf}, dynamic memory
allocation within the \gls{bpf} context is impossible. These dummy maps offer a way to access
additional memory from a pool reserved at the time the map was loaded into the kernel.

\subsubsection*{eBPF Program Types and Use Cases}

Each \gls{ebpf} program has a specific \textit{program type}, which determines both the set of
system events to which the program can attach and the set of allowed kernel helpers that
can be called from within the program context. Each program type roughly corresponds with
a distinct \gls{ebpf} use case. For the purposes of this thesis, we will primarily be dealing
with \textit{\gls{lsm} probes}, \textit{raw tracepoints}, \textit{fentry/fexit probes}, and
\textit{uprobes/uretprobes}, as they form the basis of \bpfbox{} and \bpfcontain{}'s
kernelspace implementations. \Cref{tab:program-types} summarizes the relevant program types
and their properties.

\begingroup\footnotesize
\begin{longtable}[c]{lp{4.2in}}
\caption[A selection of relevant eBPF program types for \bpfbox{} and \bpfcontain{}]{A selection of relevant \gls{ebpf} program types for \bpfbox{} and \bpfcontain{}.}%
\label{tab:program-types}\\
  \toprule
  Program Type & Description\\
  \midrule
  \textit{\gls{lsm} Probes}    & \gls{lsm} probes~\cite{singh2019_krsi} attach to the kernel's \gls{lsm} hooks and can be used to audit security events and make policy decisions.\\
  \textit{Raw Tracepoints}     & Raw tracepoint programs attach to a stable tracing interface exposed by the Linux kernel. Tracepoints are considered a stable API, but are more limiting than alternatives such as Kprobes or Fentry probes.\\
  \textit{Kprobes/Kretprobes}  & Kprobe programs can attach to any kernel function, by replacing the function with a trap into the \gls{bpf} program. The \gls{bpf} program has read-only access to the function arguments. Kretprobes work in the same way, but handle function returns instead of function calls.\\
  \textit{Fentry/Fexit Probes} & A more efficient version of Kprobes and Kretprobes that directly trampolines into the \gls{bpf} program instead of trapping. These programs can also be used to modify the return value of specifically allowlisted kernel functions (e.g.~system call implementations).\\
  \textit{Uprobes/Uretprobes}  & The userspace equivalent of Kprobes and Kretprobes.\\
  \bottomrule
\end{longtable}
\endgroup

\subsubsection*{LSM Probes: Making Security Decisions with eBPF}

It is worth spending more time focusing specifically on \gls{lsm} probes, as these are used
extensively in \bpfbox{} and \bpfcontain{} to enforce policy over security-sensitive
events. Introduced by KP Singh in his KRSI (Kernel Runtime Security Instrumentation)
patch~\cite{singh2019_krsi}, \gls{lsm} probes define a canonical framework for attaching \gls{ebpf}
programs to the Linux kernel's \gls{lsm} security hooks (c.f.~\Cref{ss:mac}). Unlike
traditional \gls{lsm}s which are implemented as static kernel modules, \gls{lsm} probes are
\textit{dynamically attachable}, meaning that \gls{mac} and audit policy can be adjusted at
runtime, simply by loading a new \gls{ebpf} program.  \Cref{fig:bpf-lsm} depicts how \gls{lsm} probes
integrate with the \gls{lsm} framework.

\begin{figure}[tbp]
  \centering
  \includegraphics[width=0.8\linewidth]{figs/background/bpf-lsm.pdf}
  \caption[How eBPF LSM probes make policy decisions]{A simplified example of how \gls{ebpf} \gls{lsm} probes make policy decisions. Privileged userspace processes can attach one or more \gls{lsm} probes to a given hook. When a userspace process requests a privileged operation, the kernel implicitly calls into the corresponding \gls{lsm} hooks, which in turn invoke the logic associated with each \gls{lsm}. A shim \gls{lsm} is responsible for invoking each \gls{lsm} probe, and any resulting policy decisions are taken together to arrive at a final decision. As with ordinary \gls{lsm}s, the final decision is consensus-based. That is, if \textit{any \gls{lsm}s} or \textit{any \gls{bpf} \gls{lsm} probes} disagree on a policy decision, the privileged operation is denied.}%
  \label{fig:bpf-lsm}
\end{figure}

As with other LSMs, LSM probes implement a form of mandatory access control. Each LSM
probe can be attached to one or more LSM hooks defined in the kernel. When the hook fires
(i.e.~when a task requests a privileged operation form the kernel), every attached probe
fires as part of the normal LSM pipeline. The body of the \gls{bpf} program defines filtering
and audit logic, optionally accessing maps to store and query persistent state. The \gls{bpf}
program then returns a security decision about whether the requested operation should be
allowed or denied.  In order for an operation to be allowed, \textit{all} other LSMs and
LSM probes must agree on the policy decision and ordinary security checks performed by the
operating system must also succeed. In other words, it is not possible to grant additional
privileges using an LSM probe.

Owing to the properties discussed earlier in this section, \gls{ebpf} confers a natural
flexibility to LSM probes quite unlike that of traditional LSM-based security frameworks.
In particular, LSM probes can be attached at runtime and can cooperate with other \gls{ebpf}
program types using \gls{ebpf} maps (c.f.~\Cref{ss:bpf-maps-bg}). This notion of cooperating
programs presents an opportunity to design modular policy enforcement mechanisms that
operate beyond the scope of the LSM hooks framework itself.  Another key advantage of LSM
probes over traditional LSMs lies in their adoptability.  While industry actors may be
understandably reluctant to adopt \enquote{yet another out-of-tree LSM}, a security
mechanism based on \gls{ebpf} does not carry the same technical baggage.  \gls{ebpf} programs are safe
to use in production and can be deployed at runtime on an unmodified kernel.  This makes
\gls{ebpf} a particularly attractive target for developing new security solutions.

\subsection{eBPF Maps}%
\label{ss:bpf-maps-bg}

\gls{ebpf} maps serve as both a runtime data store for \gls{ebpf} programs and the canonical method of
communication between \gls{ebpf} programs and other \gls{ebpf} programs, and \gls{ebpf} programs and
userspace applications. Like \gls{ebpf} programs, maps can be pinned to \gls{bpffs} to
increment their reference count in the kernel. Concurrent access to \gls{ebpf} maps from within
kernelspace is protected by an implicit RCU lock, and a spinlock concurrency primitive is
exposed via a helper function to guard map accesses between kernelspace and userspace.
From the \gls{ebpf} side, maps can be accessed using a set of provided helper functions.
Userspace applications can access maps using the \texttt{bpf(2)} system call or through
direct memory-mapping (only available for arrays) via
\texttt{mmap(2)~\cite{gregg2019_bpf}}. While many \gls{ebpf} maps are designed to be generic,
others are highly specialized for specific use cases. \bpfcontain{} and \bpfbox{} make
use of several \gls{ebpf} map types, which are summarized in \Cref{tab:map-types}.

\begingroup\footnotesize
\begin{longtable}[c]{lp{3.9in}}
\caption[A selection of relevant eBPF map types for \bpfbox{} and \bpfcontain{}]{A selection of relevant \gls{ebpf} map types for \bpfbox{} and \bpfcontain{}.}%
\label{tab:map-types}\\
  \toprule
  Map Type & Description\\
  \midrule
  \textit{\gls{bpf} Hashmap}           & A key-value hashmap. Keys and values can be arbitrary data structures.\\
  \textit{\gls{bpf} Array}             & A fixed-size array with integer indices. Values can be arbitrary data structures.\\
  \textit{\gls{bpf} Array/Map of Maps} & A \gls{bpf} array or map that stores handles into \textit{other maps}.\\
  \textit{\gls{bpf} Per-CPU Array/Map} & Like a \gls{bpf} hashmap or \gls{bpf} array but with a separate copy per logical CPU\@. This enables concurrent access across CPUs, but without synchronization.\\
  \textit{\gls{bpf} Local Storage}     & A dummy \gls{bpf} map that provides a handle into local storage for a given kernel data structure. For instance, task local storage provides storage per-task-struct. Values can be arbitrary data structures.\\
  \textit{\gls{bpf} Ringbuf}           & A concurrent circular buffer that passes event handles from kernelspace to userspace. To communicate, \gls{ebpf} programs submit events and userspace applications poll events.\\
  \bottomrule
\end{longtable}
\endgroup

\subsection{Userspace Front-Ends}%
\label{ss:bpf-userspace}

Although \gls{ebpf} programs and maps can exist on their own after being pinned to
\gls{bpffs}, the more common approach is to manage their lifetime using a controlling
process. Userspace applications implementing such a controlling process typically use an
\gls{ebpf} front-end framework to facilitate loading and interacting with programs and maps.
A number of such front-ends exist~\cite{gobpf, bcc, libbpf, libbpf-rs, libbpfgo,
cilium-ebpf, redbpf}, some more practical than others.
\textit{bcc}~\cite{bcc} was the first \gls{ebpf} framework to offer high-level userspace tooling
around \gls{ebpf}, providing an LLVM backend for compiling \gls{ebpf} programs and a Python library
for loading and interacting with them. \textit{libbpf}~\cite{libbpf} offers a pure
C alternative to bcc and has since been upstreamed into the Linux kernel.
\textit{libbpf-rs}~\cite{libbpf-rs} and \textit{libbpfgo}~\cite{libbpfgo} offer Rust
and Golang bindings for libbpf respectively. Other tooling~\cite{cilium-ebpf, redbpf}
bypasses libbpf entirely, providing fully native \gls{ebpf} bindings for Rust and Golang.

\subsubsection*{Libbpf and BPF CO-RE}%

Among the myriad of userspace front-ends available for \gls{ebpf}, libbpf stands out as the only
one with official upstream support from the Linux kernel. Recent improvements to libbpf
have solidified its position as the dominant framework. In particular, libbpf supports
a new way of compiling a loading \gls{bpf} programs into the kernel, \gls{bpf}
\gls{core}~\cite{gregg2020_core, nakryiko2020_core}. \gls{bpf} \gls{core} uses \gls{btf}
debugging information exposed by the kernel, along with load-time relocation logic to
support loading the same compiled \gls{ebpf} bytecode across multiple target kernels. The
precise technical details of how \gls{core} works are beyond the scope of this thesis.

With libbpf and \gls{core}, \gls{ebpf} programs can now be compiled once and run on any target kernel
that supports the required \gls{bpf} features. This provides a powerful advantage over other
\gls{ebpf} frameworks and even alternatives to \gls{ebpf}, such as loadable kernel modules. A \gls{core}
program that runs on one kernel will be guaranteed to run on another of the same version
or higher, barring any API incompatibilities like changes in a hooker function signature.
Such incompatibilities can be resolved with the use of built-in kernel configuration
checks.

\bpfcontain{} (c.f.~\Cref{c:bpfcontain}) leverages libbpf and \gls{core} through libbpf-rs, the
canonical Rust bindings for libbpf, providing adoptability advantages over the original
\bpfbox{} prototype (c.f.~\Cref{c:bpfbox}), which uses bcc.

\subsection{Comparing eBPF with Loadable Kernel Modules}%
\label{ss:bpf-vs-modules}

Before \gls{ebpf}, the primary means of modifying the Linux kernel at runtime was through the
use of \textit{loadable kernel modules}~\cite{corbet1998_device_drivers}. A kernel module
can be thought of as a discrete bundle of code that can be loaded into the kernel (or
compiled into its binary image). Like other kernel code, including \gls{ebpf}, modules are
event-based and run in ring 0, responding to and handling system events as they occur.
Since kernel modules and \gls{ebpf} can serve similar (but not strictly equivalent) purposes,
comparing the two can offer some insight about how they differ and which technology is
better fit for a specific purpose.

At a first approximation, \gls{ebpf} differs from kernel modules in the following meaningful ways~\cite{gregg2019_bpf}:
\begin{enumerate}
  \item \gls{ebpf} programs \textbf{must pass verification checks} before they can be loaded into the
  kernel. This verification step provides assurances about program safety. For instance, \gls{ebpf}
  programs are guaranteed not to deadlock the kernel, and are far less likely to suffer from
  memory safety issues. In contrast, misuse of kernel APIs in a kernel module can have dangerous
  implications for system safety and security.

  \item An implicit advantage provided by \gls{ebpf} is that \gls{bpf} programs can be \textbf{easier
  to reason about} than other kernel code. \gls{ebpf} abstracts away much of the complex
  functionality required to make kernel code operate correctly by providing implicit
  guarantees about program execution. Even helper functions, which offer functionality
  beyond the scope of verifiability, must obey a predetermined contract with the verifier
  in order to be considered safe. Thus, when an \gls{ebpf} program passes verification, there is
  a much higher likelihood that it will \enquote{just work.}

  \item \gls{ebpf} \textbf{exposes map-like data structures} to facilitate runtime data storage,
  communication between \gls{ebpf} programs, and communication with userspace applications. In
  the case of kernel modules, data structures often must be implemented by hand, taking
  great care not to introduce potential bugs or security vulnerabilities, particularly in
  the case of memory management. Communication with userspace from a kernel module might
  be done via netlink sockets, file operations, or similar
  means~\cite{corbet1998_device_drivers}. These modes of communication are often less
  streamlined and, in the case of file operations, must be implemented by hand, increasing
  the likelihood of programmer error.

  \item \gls{ebpf} programs are \textbf{not Turing-complete}. Intuitively, this means that
  the set of operations a kernel module can perform is a strict superset of \gls{ebpf}. While
  this may appear to be a hugely limiting factor, in practice \gls{ebpf} programs are often
  sufficient to implement sophisticated tracing, filtering, and policy enforcement logic.
  Where the verifier gets in the way, the programmer can reach for a number of helper
  functions provided by the kernel to achieve more complex behaviour.

  \item \gls{ebpf} is \textbf{not suitable for implementing device drivers} or other complex
  functionality that requires ad-hoc access to various kernel facilities and write access
  to arbitrary memory locations.
\end{enumerate}

In summary, \gls{ebpf} is useful for observability use cases, or cases in which the functional
requirements of the kernel code are not expected to be complex or might be expected to
change frequently. \gls{ebpf} programs and maps are particularly good at separation of concerns,
composability, and modularity. \gls{ebpf} maps facilitate easy communication between kernelspace
and userspace, and provide the ability to build relationships between data from different
program types. Kernel modules should be preferred for the implementation of more complex
kernel functionality, such as device drivers.


\chapter{The Confinement Problem}%
\label{c:confinement-problem}
Researchers have been studying confinement for decades~\cite{lampson1973_confinement}, and
have been designing and applying confinement primitives since the early days of
time-sharing computers and multi-tenant systems~\cite{shu2016_security_isolation_study}.
Despite decades of research into the confinement problem, the status quo of confinement on
Linux, particularly from the perspective of containers, is in a sorry state. This chapter
outlines the difficulties that arise in the Linux confinement ecosystem due to its
inherent complexity, inflexible and interdependent components, and the difficulties that
arise in adopting newly-proposed alternatives. These insights represent the key
motivating factors behind the design of \bpfbox{} and \bpfcontain{} and can potentially
inform the way we think about new confinement mechanisms going forward.

\section{Rethinking the Virtualization Narrative}%

\begin{inprogress}
  \begin{itemize}
    \item Just like the word container makes people think about security
    \item Type I and Type II hypervisor, the way it's depicted, makes people think something else is happening
    \item The representation and terminology
    \item This makes sense to talk about it if we connect it to containers
    \item Why can't containers give as good or better security than hypervisors
    \item Can we just get our enforcement clear
    \item Virtual machines seem like they define a clear boundary, but it isn't actually
          so clear in practice because all these things are being shared across boundaries
    \item People think of virtual machines as being inherently more secure, the security
          benefit is more of an obfuscation thing related to the semantic gap but with the right
          tools you can still cross it freely
    \item VMs have all these little holes everywhere and the policy is not centralized\,---\,we
          have policy through mechanism
    \item There's no reason why containers can't be as (if not more) secure than virtual machines
    \item Because we can make this boundary very clear
    \item \bpfcontain{} can be seen as a step towards this
  \end{itemize}
\end{inprogress}

\section{Fundamental Issues with Linux Confinement}%
\label{s:cp-confinement-issues}

\begin{enumerate}
  \item \textbf{Complexity and Interdependence.}
    Existing confinement primitives are overly complex and designed for use
    cases beyond simple process confinement. This results in a pattern of
    abusing existing mechanisms
  \begin{inprogress}
    \begin{itemize}
      \item Existing confinement solutions are overly complex
      \item Based on a number of low-level primitives that were originally designed for
            totally different tasks.
      \item Namespaces were designed to virtualize resources. They provide a form of
            isolation but not confinement; need a way to deal with namespace escapes
      \item Cgroups, similarly, were designed to virtualize the availability of system
            resources, not to directly confine.
      \item Unix \gls{dac} is far too coarse-grained and easy to bypass to be practically useful for confinement.
      \item POSIX capabilities can be used to reduce overprivilege by portioning root privileges, but do not
            implement confinement by themselves.
      \item Seccomp-bpf works well to reduce the availability of system calls, but writing
            classic \gls{bpf} filters is complex and error-prone. Anything beyond basic system call
            filtering quickly becomes untenable, particularly considering race conditions when
            checking arguments.
    \end{itemize}
  \end{inprogress}

  \item \textbf{Inappropriate Defaults.}
  \begin{inprogress}
    \begin{itemize}
      \item
    \end{itemize}
  \end{inprogress}

  \item \textbf{Difficulty Adopting New Solutions.}
  \begin{inprogress}
    \begin{itemize}
      \item Motivated by the above difficulties, academics are often tempted to propose new confinement solutions.
      \item Many try to solve the problem by simply recombining and reusing these existing primitives in new ways.
      \item This isn't really a step forward, as we are still subject to limitations introduced in item 1 and item 2.
      \item To really solve the problem, we need kernel-level support for something new.
      \item The issue is that new solutions based on kernel support are not necessarily
            adoptable. New kernel code can introduce bugs and security vulnerabilities, and
            needs to be thoroughly tested before it can be considered production-ready.
      \item Another problem arises when we consider container-specific confinement as an
            end goal; not everyone can agree on what the definition of a \enquote{container}
            even is, so how can we hope to reach agreement on what a new abstraction for
            container-based confinement would even look like.
      \item To solve this problem, we need a way to add new abstractions to the kernel in
            a way that is neither binding nor limited by the lack of adoptability associated
            with traditional kernel-based solutions.
    \end{itemize}
  \end{inprogress}
\end{enumerate}




\section{Confinement in Container Management Frameworks}%
\label{s:cp-containers}

\begin{inprogress}
Linux containers have three broad goals. In order of perceived prioritization in existing
container implementations, they are:
\begin{enumerate}
  \item \textbf{Dependency Management / Reproducibility.}
    Containers should provide an easy and robust framework for creating reproducible
    development environments. Dependencies should be maximally self-contained such that
    a containerized environment \enquote{just works} to the maximum possible extent. We
    can see examples of this property in Docker, the predominant container framework to
    date. Docker Hub~\cite{docker_hub} allows container images to be pulled from the
    Internet, recombined, and used to create further images. The end result is a flexible
    framework for creating and distributing reproducible development environments.

  \item \textbf{Virtualization.}
    Containers should virtualize system resources, creating the illusion of running on
    a separate physical machine. Where possible, resources should be transparently reused
    between multiple containers (e.g.~sharing a single base copy of the same shared
    library between two container images). To achieve virtualization, containers generally
    rely on the namespaces and cgroups primitives provided by the Linux kernel. Overlay
    filesystems~\cite{overlayfs} combined with the mount namespace aid containers one-way
    sharing of filesystem resources.

  \item \textbf{Confinement.}
    Containerized processes should be confined by default. That is, a containerized
    process should have access to the minimal set of privileges required for it to operate
    normally. The extent to which this property is achieved by conventional container
    frameworks varies greatly, both by framework and by individual
    deployment~\cite{sultan2019_container_security, lin2018_container_security,
    bui2015_docker_analysis}.
\end{enumerate}

Unfortunately, the aforementioned goals are not only ordered by their decreasing
prioritization in extant container management frameworks, but they are also ordered by
increasing relevance to container security. That is to say, existing frameworks generally
prioritize goals unrelated to security and leave security as an afterthought. Since
containers are necessarily less isolated from the host system than virtual
machines~\cite{sultan2019_container_security, eder2016_hypervisor_container}, one might
expect container security to be of paramount importance. This motivates the need to
revisit container security and approach it from a confinement-first perspective.
\end{inprogress}

\begin{inprogress}
  Container security policies are often overly-generic and ill-suited to fine-grained
  confinement. To achieve confinement in the first place, container frameworks cobble
  together existing confinement technologies and apply them in confusing and
  difficult-to-audit ways, overlapping and recombining default and generated policies. The
  end result is a complex security soup with little room for policy customization or
  auditability. \todo{Examples here}

  Even worse, many container management systems operate under a fail-open approach when the
  necessary security mechanisms are not supported. This results in low-security deployments,
  often without even notifying the user that there may be such a configuration. Since the
  end-user generally doesn't even participate in the policy authorship process, they may not
  even be aware of the level of protection that is being applied, resulting in a dangerous
  false sense of security. \todo{Examples here}
\end{inprogress}




\section{Design Goals}%
\label{s:design-goals}

Using the above analysis of the confinement problem, we can derive a clear set of design
goals for \bpfbox{} and \bpfcontain{}, such that they approach a solution to issues that
plague the status quo. In particular, we derive the following three design principles.
Note that these design principles are each the polar opposite of the three major problems
identified at the beginning of this chapter (c.f.~\Cref{s:cp-confinement-issues}).

\begin{enumerate}
  \item \textbf{Simplicity and Flexibility.}

  \item \textbf{Sensible Defaults.}

  \item \textbf{Adoptability.}
\end{enumerate}



\section{The \bpfbox{} and \bpfcontain{} Threat Model}%
\label{s:threat-model}


\chapter{\bpfbox: A Prototype Process Confinement Mechanism}%
\label{c:bpfbox}
This chapter presents the design and implementation of \bpfbox{}, an initial research
prototype of an \gls{ebpf}-based confinement framework. \bpfbox{} is the first
full-fledged confinement framework to leverage \gls{krsi}'s \gls{lsm} programs to enforce
high-level policy. Using \gls{ebpf}, it combines various behavioural aspects of the
sandboxed application from both userspace and kernelspace to enforce a simple, yet
fine-grained policy defined in a domain-specific policy language. This chapter was a part
of a previously published paper at CCSW'2020, co-authored with Anil Somayaji and David
Barrera~\cite{findlay2020_bpfbox}.



\section{\bpfbox{} Overview}

At a high level, \bpfbox{} is a confinement mechanism based on \gls{ebpf}. \todo{Outline basic components}

\todo{Describe how \bpfbox{} enforces policy from start to finish, refer to \Cref{fig:bpfbox-overview}}

\begin{figure}[htpb]
  \centering
  \includegraphics[width=0.8\linewidth]{figs/bpfbox/overview.pdf}
  \caption[A high-level overview of how \bpfbox{} confines applications]{
    A high-level overview of how \bpfbox{} confines applications. Users write policy files
    which the daemon encodes into \gls{ebpf} maps. The daemon loads these maps, along with
    policy enforcement and tracing programs into the kernel. At runtime, \bpfbox{}'s
    \gls{bpf} programs trace the application and enforce policy.
  }%
  \label{fig:bpfbox-overview}
\end{figure}

\bpfbox{}'s fundamental enforcement strategy is based on several \gls{bpf} program and map
types, summarized as follows. The reader is encouraged to revisit
\Cref{ss:bpf-programs-bg,ss:bpf-maps-bg} in \Cref{c:background} for clarification on
specific programs and maps.

\begin{itemize}
  \item \textbf{State Maps} are \gls{ebpf} hash maps that associate a kernel \gls{tgid}
  with a specific task state. The task state holds information including policy
  association, task liveness, and other metadata used for enforcement.

  \item \textbf{Policy Maps} are \gls{ebpf} hash maps that encode \bpfbox{} policy for
  various categories of access. Each map encodes an access vector and policy ID pair that
  is mapped to the corresponding enforcement decision. At runtime, \bpfbox{} queries these
  maps before making an enforcement decision on a specific access pattern.

  \item \textbf{Tracepoints} enable \bpfbox{} to track the state of a process from the
  point where it forks or executes a new binary to when it exits. \bpfbox{} stores
  per-process state from its tracepoints in \textit{state maps} for later use.

  \item \textbf{\gls{lsm} Probes} enforce policy by attaching to \gls{lsm} hooks in the
  kernel. These hooks are called by kernel functions such as system call implementations
  and trigger the corresponding \gls{bpf} program, which enforces a policy decision on the
  target application. To enforce policy, \bpfbox{}'s \gls{lsm} probes query \textit{policy
  maps} and \textit{state maps}.

  \item \textbf{Kprobes and Uprobes} are used to enforce \textit{stateful policy},
  according to what function calls a process has made, in kernelspace and userspace
  respectively. A \bpfbox{} policy file may outline that certain rules should only apply
  within the context of a specific function call; when a process runs some code that
  results in such a function call, the corresponding kprobe or uprobe will make an update
  to the process' \textit{state map} to indicate this. \bpfbox{} then considers this state
  when making later enforcement decisions.
\end{itemize}


\section{\bpfbox{} Implementation}

\todo{This section will present the implementation details of \bpfbox{}, taken from our paper.}




\section{\bpfbox{} Policy Language}

\todo{This section will present and document the policy language of \bpfbox{}, taken from our paper.}





\section{Limitations and the Transition Toward \bpfcontain{}}

\todo{This section will discuss limitations of \bpfbox{}, how it was a rough first-cut at
a solution to the confinement problem, and talk about the aspects of \bpfbox{} that
\bpfcontain{} improves upon.}

\begin{inprogress}
\begin{itemize}
  \item
\end{itemize}
\end{inprogress}


\chapter{\bpfcontain: Extending \bpfbox{} to Model Containers}%
\label{c:bpfcontain}


\section{\bpfcontain{} Overview}

\todo{This section will present an architectural overview of \bpfcontain{}, summarize key
features and what it can do.}

\section{Extending \bpfbox{} to Model Containers}

\todo{This section will discuss how \bpfcontain{} extends \bpfbox{} to model containers.
Specifically, the idea is to enforce policy at the granularity of an entire container
rather than an individual process. This lets us get away with all sorts of implicit
policy\,---\,all operations within the confines of the container that do not affect the
rest of the system are permitted. Operations that impact the rest of the system, such as
those that modify kernel code, system parameters, or similar are denied. Everything else
can be specified as a rule. This basically allows us to get away with almost no policy
language whatsoever. A nice way to put it: \enquote{The policy language is used to define
the exceptions rather than the rules}.}



\section{\bpfcontain{} Implementation}

\todo{This section will present the implementation details of \bpfcontain{}, taken from our paper.}




\section{\bpfcontain{} Policy Language}

\todo{This section will present and document the policy language of \bpfcontain{}, taken from our paper.}




\section{Implicit Policy Under \bpfcontain{}}

\todo{This section will describe \bpfcontain{}'s implicit policy in more detail and describe the rationale behind the approach.}






\todo{How do we conclude this chapter?}


\chapter{Evaluation}%
\label{c:evaluation}
This chapter presents an evaluation of \bpfbox{} and \bpfcontain{} in terms of their
performance and security. \Cref{s:eval-performance} presents the methodology and results
of a performance evaluation involving micro- and macro-benchmarking of \bpfbox{} and
\bpfcontain{}. Results are compared with AppArmor~\cite{cowan2000_apparmor}, a popular
\gls{lsm} framework for \gls{mac} security policy. Finally, \Cref{s:eval-security}
presents a security analysis of \bpfbox{} and \bpfcontain{} under the threat model
outlined in \Cref{s:cp-threat-model} of \Cref{c:confinement-problem}.

\begin{inprogress}
  Might be useful for later when we compare with AppArmor:
  \begin{itemize}
    \item Zhang \etal~\cite{zhang2021_lsm_file_overhead} evaluated the performance
    overhead of LSMs on file system operations; they mentioned (really damning) performance
    statistics for SELinux that can be compared with BPFContain and AppArmor
  \end{itemize}
\end{inprogress}

\section{Performance Evaluation}%
\label{s:eval-performance}

\begin{inprogress}
Here, we evaluate the overhead of \bpfbox{} and \bpfcontain{} by examining their performance
over a series of macro- and micro-benchmarks. For
\end{inprogress}

\subsection{Methodology}%
\label{ss:performance-methodology}



\section{Security Analysis}%
\label{s:eval-security}



\section{Summary}%
\label{s:eval-summary}


\chapter{Case Studies}%
\label{c:case-studies}
In this chapter, we examine specific case studies, applying \bpfbox{} and \bpfcontain{}
policies to solve realistic problems. In particular, we examine the default Docker policy
and a more complex example confining a web server and database. We also provide an example
of how \bpfcontain{} can be used to apply basic confinement to an untrusted container. To
offer a basis for comparison, we contrast presented policies with some available
equivalents and discuss how the semantics of the policy language and enforcement engine
can impact the resulting policy file.

% \section{Methodology}

% \todo{This section will present the methodology used to select existing policies and
% compare them with \bpfbox{} and \bpfcontain{} policies}


\section{The Default Docker Policy}

Docker~\cite{docker_security} applies a coarse-grained default confinement policy to all
containers using a combination of Linux confinement primitives. On supported
systems\footnote{Recall that not all Linux distributions support AppArmor or Seccomp-bpf
to begin with. In such cases, Docker simply discards its default confinement policy
altogether.}, this includes a default AppArmor policy template~\cite{docker_apparmor,
docker_default_apparmor}, a default Seccomp-bpf profile, and a set of POSIX capabilities
which are dropped at runtime~\cite{docker_security}.

Docker's policy defaults are highly coarse-grained, with an emphasis on practical security
while ensuring that the vast majority of container configurations will \enquote{just
work,} out of the box. This affords a practical opportunity to examine how \bpfbox{} and
\bpfcontain{} policies compare with the default Docker policy. \Cref{tab:docker-default}
summarizes the key aspects of Docker's confinement policy, highlighting default access
levels enforced by various Linux confinement primitives. \Cref{lst:docker-default} depicts
Docker's default AppArmor template, taken directly from the Docker sources on
GitHub~\cite{docker_default_apparmor}.

\begin{table}[htpb]
  \centering
  \caption[The default Docker confinement policy]{
    A summary of Docker's default confinement policy~\cite{docker_security,
    docker_apparmor, docker_default_apparmor}. Policy is enforced using a number of Linux
    confinement primitives, including AppArmor, Seccomp-bpf, and dropped POSIX
    capabilities at runtime. Docker generates and loads AppArmor policy at container
    runtime using a pre-determined, coarse-grained AppArmor template file
    (c.f.~\Cref{lst:docker-default}).
  }%
  \label{tab:docker-default}
  \footnotesize
  \begin{tabular}{lp{2in}p{1.6in}}
  \toprule
  Access Category & Default & Docker Implementation \\
  \midrule
  Files & Allow access to all files except specific procfs and sysfs entries. & AppArmor Template \\
  Filesystem Mounts & Deny all filesystem mounts. & AppArmor Template \\
  POSIX Capabilities & All capabilities enabled in AppArmor.  Drop specific capabilities at runtime. & AppArmor Template and Dropped Capabilities \\
  Ptrace & Allowed within container. & AppArmor Template \\
  Signals & Allowed within container. & AppArmor Template \\
  Network & Allow all network access. & AppArmor Template \\
  \gls{ipc} & Allow all \gls{ipc} access. & AppArmor Template \\
  System Calls & Deny about 60 obsolete/dangerous system calls. & Seccomp-bpf \\
  \bottomrule
  \end{tabular}
\end{table}

\begin{lstlisting}[language=none, gobble=4,
  caption={[Docker's default AppArmor template]
    Docker's default AppArmor template~\cite{docker_default_apparmor}, at the time of
    writing this thesis. Docker uses Go's string templating syntax to modify the AppArmor
    profile according to the current Docker version and container metadata.
  },
  label={lst:docker-default}, float]
    {{range $value := .Imports}}
      {{$value}}
    {{end}}
    profile {{.Name}} flags=(attach_disconnected,mediate_deleted) {
    {{range $value := .InnerImports}}
      {{$value}}
    {{end}}
      network,
      capability,
      file,
      umount,
    {{if ge .Version 208096}}
      # Host (privileged) processes may send signals to container processes.
      signal (receive) peer=unconfined,
      # dockerd may send signals to container processes (for "docker kill").
      signal (receive) peer={{.DaemonProfile}},
      # Container processes may send signals amongst themselves.
      signal (send,receive) peer={{.Name}},
    {{end}}
     # deny write for all files directly in /proc (not in a subdir)
      deny @{PROC}/* w,
      # deny write to files not in /proc/<number>/** or /proc/sys/**
      deny @{PROC}/{[^1-9],[^1-9][^0-9],
        [^1-9s][^0-9y][^0-9s],[^1-9][^0-9][^0-9][^0-9]*}/** w,
      # deny /proc/sys except /proc/sys/k* (effectively /proc/sys/kernel)
      deny @{PROC}/sys/[^k]** w,
      # deny everything except shm* in /proc/sys/kernel/
      deny @{PROC}/sys/kernel/{?,??,[^s][^h][^m]**} w,
      deny @{PROC}/sysrq-trigger rwklx,
      deny @{PROC}/kcore rwklx,
      deny mount,
      deny /sys/[^f]*/** wklx,
      deny /sys/f[^s]*/** wklx,
      deny /sys/fs/[^c]*/** wklx,
      deny /sys/fs/c[^g]*/** wklx,
      deny /sys/fs/cg[^r]*/** wklx,
      deny /sys/firmware/** rwklx,
      deny /sys/kernel/security/** rwklx,
    {{if ge .Version 208095}}
      # suppress ptrace denials when using 'docker ps' or using 'ps' inside a container
      ptrace (trace,read,tracedby,readby) peer={{.Name}},
    {{end}}
    }
\end{lstlisting}

\subsubsection{\bpfbox{}}

We begin by examining a mostly equivalent policy in \bpfbox{}, given in
\Cref{lst:bpfbox-docker-default}.  Re-implementing Docker's default confinement policy in
\bpfbox{} is surprisingly challenging. \bpfbox{} is not designed to implement
coarse-grained confinement policy, and so specifying things like global access to all
files is impossible. We compromise by granting recursive access to all files within
a given filesystem, repeating the process for each filesystem as required. This is
\textit{not} the intended use case for \bpfbox{} file rules, but it is required to match
the over-permissive filesystem access provisioned by Docker. Aside from
filesystem-specific policy, most of Docker's default policy can be implemented relatively
easily and cleanly in \bpfbox{}'s policy language.

\begin{lstlisting}[language=bpfbox, gobble=4,
  caption={[Implementing the default Docker policy in \bpfbox{}]
    Implementing the default Docker policy in \bpfbox{}.
    \todo{High-level overview of the policy}
  },
  label={lst:bpfbox-docker-default}]
    #![profile "/path/to/init/program"]

    #[allow] {
      /* Allow essentially global access to a filesystem */
      fs("/path/to/filesystem/**", read|write|setattr|getattr|rm|link|ioctl)
      /* Repeat for others... */

      /* Allow access to /proc/sys/kernel/shm* */
      fs("/proc/sys/kernel/shm*", read|write|setattr|getattr)

      /* Sensible default access for procfs per-pid entries */
      proc(self, read|write)
      proc(child, read|write)
    }

    #[allow]
    #[taint]
    {
      /* Access to network families */
      net(inet, any)
      net(inet6, any)
      net(unix, any)

      /* Ptrace child processes */
      ptrace(child, read|write|attach)

      /* Send sigchld up to parent processes, any signal to children */
      signal(parent, sigchld)
      signal(child, any)
    }

    #[transition]
    #[untaint]
    {
      /* Allow execve calls to allowed executables,
       * tainting and transitioning profiles when doing so */
      fs("/path/to/allowed/executable", read|exec)
      /* Repeat for others... */
    }
\end{lstlisting}

Like Docker's AppArmor policy, our \bpfbox{} policy enables access to per-pid entries in
procfs and uses \bpfbox{}'s default-deny enforcement to restrict all others. Similar logic
applies to the \texttt{/proc/sys/kernel/shm**} entries under procfs. We also grant full
networking stack access, ptrace access for child processes, and full signal access for
child processes running under the container. Since these operations have the potential to
introduce vulnerabilities from outside sources, we mark them as tainting the corresponding
process. Leveraging taintedness, the \bpfbox{} policy eliminates the need to specify
access to shared library dependencies and other artifacts of the C runtime.

For more complex container deployments that include more than a single binary, the
\bpfbox{} policy may need to specify access to alternative executables under the
container.  We do so using an individual file rule for each executable, optionally
specifying that the process should untaint itself and/or transition to a new profile.
Notably, the version of \bpfbox{} presented in this thesis does \textit{not} include
capability-level policy, and so it is not included here\footnote{\bpfcontain{} later
rectified this gap in \bpfbox{}'s policy language.}. However, the default Docker
confinement policy does not implement capability-level filtering anyway, instead relying
on dropped capabilities at runtime.

Although the \bpfbox{} policy depicted in \Cref{lst:bpfbox-docker-default} does not fully
map to the precise Docker default policy, it gets very close in most respects, aside from
filesystem policy. Under \bpfbox{}, filesystem policy is necessarily finer-grained, as it
does not support the ability to specify coarse-grained access to all files on the system.
Despite these challenges, the end-result is a functional (and, in some aspects, more
secure) alternative to the default Docker policy.

\subsubsection{\bpfcontain{}}

Having examined how \bpfbox{} can be used to implement an approximate version of Docker's
default confinement policy, we now turn our attention to \bpfcontain{}.
\Cref{lst:bpfcontain-docker-default} shows the full \bpfcontain{} policy. Note that many
aspects of Docker's default policy are covered by \bpfcontain{}'s default
container-boundary enforcement. Using this to its advantage, the \bpfcontain{} policy is
significantly simpler than both the AppArmor and \bpfbox{} versions while maintaining the
same level of expressiveness.

\begin{lstlisting}[language=yaml, gobble=4,
  caption={[Implementing the default Docker policy in \bpfcontain{}]
    Implementing the default Docker policy in \bpfcontain{}. A few coarse-grained
    allow-rules can be used to capture permissive Docker defaults that are not covered
    under \bpfcontain{}'s default policy. Other aspects of the Docker defaults are already
    covered under \bpfcontain{} defaults, such as the inability to mount filesystems,
    perform a number of privileged system calls, and interact with non-pid entries in
    procfs and sysfs. Due to \bpfcontain{}'s default policy for file access and \gls{ipc},
    it is neither necessary to specify file access rules for files within the container's
    overlay filesystem not \gls{ipc} rules for processes within the container.
  },
  label={lst:bpfcontain-docker-default}]
    name: default-docker
    defaultTaint: true

    allow:
      # Grant access to the entire root filesystem
      fs: {pathname: /, access: any}
      # Grant access to tempfs
      fs: {pathname: /tmp, access: any}
      # Grant read/write access to /proc/sys/kernel/shm*
      file: {pathname: /proc/sys/kernel/shm*, access: rw}

      # Grant access to the entire networking stack
      net: any

      # Enable Docker default capabilities
      # All other capabilities are denied
      capability:
        - chown
        - dacoverride
        - fsetid
        - fowner
        - mknod
        - netraw
        - setgid
        - setuid
        - setfcap
        - setpcap
        - netbindservice
        - syschroot
        - kill
        - auditwrite
\end{lstlisting}

Compared with \bpfbox{}, the \bpfcontain{} version of Docker's default policy is
significantly simpler and fits more cleanly with Docker's AppArmor policy. This
improvement is a direct result of a number of critical differences between \bpfbox{} and
\bpfcontain{}. Whereas \bpfbox{} was designed for fine-grained, process-level confinement,
\bpfcontain{} was directly designed with containers in mind. Since \bpfcontain{} policies
are designed to be container-specific, they are far more appropriate for a use case
centered around the confinement of containers. In particular, \bpfcontain{} incorporates
container semantics into its default policy enforcement, greatly simplifying the resulting
policy.  Further, changes to \bpfcontain{}'s policy language, including the introduction
of a coarser-grained filesystem rule and capability rules enables the resulting policy to
more closely match the original Docker AppArmor policy.

To match Docker's default allow on all filesystem access, the \bpfcontain{} policy
includes a rule to enable any file operation on files within the root filesystem.  As with
\bpfbox{}, the point here is to match Docker's default policy, without considering the
security implications of granting full access to the entire root filesystem. We include
another rule to enable similar access on the temporary filesystem. Despite the coarse
granularity of these filesystem rules, \bpfcontain{} maintains a critical advantage over
\bpfbox{} and the original Docker policy. Due to its container-specific policy defaults,
we can achieve Docker's fine-grained protection over procfs and sysfs for free. Thus,
\bpfcontain{} entirely obviates the need to specify such rules in the policy.

As with the procfs and sysfs policy, \bpfcontain{} also includes sensible defaults for
\gls{ipc} and ptrace access. In particular, processes running within the same container
are free to perform \gls{ipc} with one another and ptrace one another, so long as the
basic Unix access rights are satisfied (e.g.~the process possesses CAP\_PTRACE or is the
direct ancestor of the tracee). In the case of signals and ptrace, these defaults directly
match the Docker policy (c.f.~\Cref{tab:docker-default}).  In other cases, these defaults
are more secure than the Docker policy, while permits all other forms of \gls{ipc}
regardless of container membership.

To prevent a container from escaping confinement or interfering with the host,
\bpfcontain{} prohibits the container from mounting filesystems, loading kernel modules,
using \gls{ebpf}, changing the system time, rebooting the system, or performing a number
of other privileged operations. These defaults also match or exceed Docker's default
policy, and thus may also be omitted from the \bpfcontain{} policy.

While many aspects of \bpfcontain{}'s default enforcement closely match the default Docker
policy, \bpfcontain{}'s defaults remain strictly less permissive. For instance, the
default Docker policy mandates that \texttt{/proc/sys/kernel/shm*} be accessible to
containers, but \bpfcontain{} denies access to all procfs entries that do not belong to
a container process. We define an exception to \bpfcontain{}'s default procfs policy by
adding an explicit allow rule on this pathname. Similarly, \bpfcontain{}'s default policy
forbids network access by default, and so we must explicitly grant the container
permission to use the networking stack. Unlike Docker, \bpfcontain{} prohibits the use of
any POSIX capability that is not directly specified in the policy file. Thus, we include
an additional allow rule that mirrors the set of capabilities dropped by Docker at
runtime.

The resulting \bpfcontain{} policy implements a strict superset of Docker's default
confinement policy, despite being significantly simpler, and more centralized.  Since
\bpfcontain{} directly models the relationship between containerized processes and their
resources, we can achieve significant portions of Docker's default policy for free. In
many cases, this default enforcement is actually finer-grained than the Docker defaults.
In order to achieve the same coarse granularity as the Docker policy, we adjust the
\bpfcontain{} policy by incorporating a few additional allow rules, granting access to
specific filesystems, the networking stack, and POSIX capabilities.

\section{Confining an Untrusted Container}

We now examine perhaps the most obvious and practical use case for \bpfcontain{}:
confining and untrusted container. For instance, consider a new container image, freshly
downloaded from Docker Hub, to be used during application development or testing.  We
assume that the system administrator does not trust this container images, and wishes to
confine the resulting containers, preventing them from damaging the rest of the system,
leaking information, or performing other unwanted actions. For this purpose, we leverage
an extremely simple \bpfcontain{} policy (essentially the \enquote{Hello World} example)
and demonstrate how it can be customized to match the container's specific needs.
\Cref{lst:bpfcontain-untrusted} depicts this policy.

\begin{lstlisting}[language=yaml, gobble=4,
  caption={[Confining an untrusted container with \bpfcontain{}]
    Confining an untrusted container with \bpfcontain{}. \bpfcontain{}'s default
    enforcement policy of defining a boundary around the container enables this policy to
    be quite simple. A default-tainted policy enables container-level confinement without
    specifying \textit{any} rules whatsoever. This policy can then be adjusted as
    required, specifying file rules to provision access to volume mounts, network rules to
    enable networking, and capability rules to enable access to specific POSIX
    capabilities.
  },
  label={lst:bpfcontain-untrusted}, float]
    name: untrusted-container
    defaultTaint: true

    allow:
      # Specify full path and access for volume mounts from the host
      file: {pathname: /path/to/volume/mount, access: rw}
      # Repeat for other volume mounts as required...

      # Uncomment if the container requires networking
      # We could also define finer-grained access by replacing the "any"
      # keyword with specific socket operations
      # net: any

      # Uncomment if the container requires any capabilities
      # capability: [dacoverride, dacreadsearch, netbindservice] # etc.
\end{lstlisting}

Note that \bpfbox{} is not the correct tool for this job. Without \bpfcontain{}'s semantic
defaults, a \bpfbox{} policy would need to manually specify every single access required
for the container to function. Thus, it is impossible to design a generically-applicable
solution using \bpfbox{}. The key advantage of \bpfcontain{} here is that it can use
container semantics to infer a default protection boundary for the container. This greatly
simplifies the resulting policy while ensuring that the container is unable to access
resources outside of its protection boundary, unless otherwise specified. The resulting
\bpfcontain{} policy is just a few lines long.

We define a default-tainted \bpfcontain{} policy called \enquote{untrusted-container}.
Just this policy alone should be enough to confine a simple container. \bpfcontain{}'s
default policy would prohibit network access, the use of any POSIX capabilities, access to
any files outside of the container's overlay filesystem, and any operations that can
impact the system as a whole. This default policy prevents entire classes of attacks,
prohibiting the container from leaking outside information, changing global system
parameters, loading code into the kernel, or forming unauthorized network connections. The
reader is encouraged to revisit \Cref{fig:bpfcontain-enforcement} on page
\pageref{fig:bpfcontain-enforcement} of \Cref{c:bpfcontain} for a depiction of how
\bpfcontain{}'s default enforcement works.

While the \bpfcontain{} default policy should be sufficient for simple use cases, more
advanced container images may require some slight modification, introducing a few allow
rules to define exceptions in \bpfcontain{}'s protection boundary. For instance, suppose
the container image requires a docker volume to be mounted at runtime. To support this use
case, we define a file rule, specifying the system path to the volume mount and the
corresponding access pattern, such as \texttt{rw} for read and write access. All other
accesses to the host filesystem remain denied. If the container requires access to the
networking stack, we similarly define a net rule.  Capability rules can be used to allow
the container to use a selected subset of POSIX capabilities, assuming it already
possesses these capabilities at runtime. For example, we may wish to grant the container
the \texttt{DAC\_OVERRIDE} and \texttt{DAC\_READ\_SEARCH} capabilities to allow it to
interact with the Docker volume we specified earlier, or the \texttt{NET\_BIND\_SERVICE}
capability to allow it to bind to privileged ports.

\section{Confining a Web Server and Database}

\todo{This section will describe how \bpfbox{} and \bpfcontain{} can be used to confine an
Apache Webserver and Postgres Database configuration and then compare the resulting policy
with Snap, AppArmor, and SELinux}



\chapter{Discussion and Concluding Remarks}%
\label{c:discussion}
This chapter discusses the relevance of \bpfbox{} and \bpfcontain{}, positioning them as
novel extensions on top of the existing confinement literature. We also examine
limitations of both research systems and present opportunities for future work. Namely, we
propose ways to address current limitations, improve the \gls{ebpf} ecosystem for
confinement use cases, add features to \bpfbox{} and \bpfcontain{}, and conduct further
research on the usability of both systems.

\section{Improving the Status Quo}%
\label{s:disc-improving}

\todo{This section will discuss how \bpfbox{} and \bpfcontain{} improve upon the status
quo in confinement. Specifically, it will compare both systems with related work.}

\todo{Simple policies encourages local policy variations, which makes life harder for the
bad guys on a big scale. Thinking about diversity at the code level can be a problem
because it goes against so much of what we do. But the policy layer makes sense as a good
place to put diversity in.}

\begin{inprogress}
  \begin{itemize}
    \item
  \end{itemize}
\end{inprogress}


\section{Limitations}%
\label{s:disc-limitations}

\todo{This section will discuss the limitations of \bpfbox{} and \bpfcontain{}}

\begin{inprogress}
  Limitations related to pathname support
  \begin{itemize}
    \item Hard to refer to files from \gls{ebpf}
    \item Currently, \bpfbox{} and \bpfcontain{} translate file policy into inodes and filesystem device IDs
    \item This is a crude workaround; it has some convenient side effects for security,
    but issues arise when we want to refer to pathnames that don't exist yet
    \item It can also cause an explosion in the size of maps storing file rules, as
    globbed paths get translated into multiple rules: one for each file that matches the
    glob

    \item The difficulty working with pathnames is partially a result of a fundamental limitation of \gls{ebpf}: difficulty manipulating strings
    \item Problem generally arises from three factors, primarily related to the verifier:
    \begin{itemize}
      \item Verifier imposes a hard 512 byte limit on stack allocations (strings need to be heap-allocated, stored in an map)
      \item Verifier imposes restrictions on how programs can loop (looping needs to provably terminate, the verifier errs on the side of caution here)
      \item Helper functions can get around these restrictions, but decent string and pathname helpers are no here yet
    \end{itemize}
    \item Another fundamental issue is that support for sleepable \gls{bpf} is new and has
    not yet matured (only a small subset of \gls{lsm} programs can currently be marked
    sleepable)
    \item Linux 5.(11?) added \texttt{bpf\_d\_path}, but this is only callable from sleepable programs, a subset of \gls{lsm} programs
    \item In the current \bpfbox{} and \bpfcontain{} design, this limitation is too restrictive
    \item Luckily, it seems like the community is working towards a general solution to
    this problem (dynamic map allocation and making more program types sleepable)
    \item As the \gls{ebpf} ecosystem evolves, it may be possible to support pathnames as
    a first class citizen, removing the requirement for working with inodes and filesystem
    numbers
  \end{itemize}

  Limitations related to policy map size
  \begin{itemize}
    \item Currently, policy maps are of a fixed size
    \item It's okay to make them big, since \gls{ebpf} does support map growth up to a fixed limit
    \item But we are still limited in total map size (todo: get current figure)
    \item In our current implementation, we simply grow policy maps from userspace when the map size would be too small to fit current rules
    \item But this approach is still limited, as it doesn't support map resizing at runtime, only at load time
    \item Once sleepable \gls{bpf} matures, we can have dynamically allocated maps of
    arbitrary size at runtime (link Alexei's LKML thread), as we can now have runtime map
    allocators where is it okay to sleep on a page fault
  \end{itemize}

  Limitations related to overhead
  \begin{itemize}
    \item Currently, \bpfbox{} and \bpfcontain{} are competitive with mainstream
    confinement solutions based on \gls{lsm} (e.g.~AppArmor, see chapter 6)
    \item This competitiveness is actually an advantage, considering that \bpfbox{} and
    \bpfcontain{} can be dynamically loaded and attached to various system events. In this sense,
    we are getting increased flexibility without paying much of a cost in performance.
    \item But, \bpfbox{} and \bpfcontain{} have the potential to be far more performant than
    conventional \gls{lsm}s in one critical case: passive overhead on the rest of the system.
    \item Due to a current limitation of how KRSI works, its \gls{ebpf} \gls{lsm} hooks are
    always globally invoked, regardless of whether the target process is of interest to us or not.
    \item The current pattern looks like (Invoke Syscall $\rightarrow$ Invoke Hook
    $\rightarrow$ Invoke BPF Program $\rightarrow$ Filter Logic $\rightarrow$ Return from
    BPF Program $\rightarrow$ Return from Hook $\rightarrow$ Return from Syscall).
    \item In the future, we may be able to move the filter logic to the step
    \textit{before}  the hook is called, or at the very least before the BPF program is
    called. This would nearly eliminate any passive overhead on the unconfined parts of the system.
    \item I have a plan for this: introducing a new namespace for \gls{bpf} programs and
    maps. (Forward ref to BPF namespace and unprivileged BPF)
  \end{itemize}

  Limitations in network policy granularity
  \begin{itemize}
    \item Network policy in \bpfbox{} and \bpfcontain{} is currently very coarse-grained
    \item Only operates at the socket level, and does not considered nuanced access
    controls such as at the per-IP-address level.
    \item This means that specifying access to the network essentially gives the process or container
    access to the global network.
    \item While this is not problematic for applications that do not require network access,
    it quickly becomes an overprivilege issue for applications that do.
    \item To fix this, we can introduce protocol-level network policy, an extension which
    is possible using currently available \gls{ebpf} technology. (Forward ref to protocol-level network policy)
  \end{itemize}
\end{inprogress}


\section{Future Work and Research Directions}%
\label{s:disc-future-work}

\todo{This section will discuss opportunities for future work.}

\subsection{Policy Language Experimentation and Usability Study}

\begin{inprogress}
  \begin{itemize}
    \item
  \end{itemize}
\end{inprogress}

\subsection{The BPF Namespace and Unprivileged BPF}

\begin{inprogress}
  \begin{itemize}
    \item
  \end{itemize}
\end{inprogress}

\subsection{OCI Compliance and Docker Integration}

\begin{inprogress}
  \begin{itemize}
    \item
  \end{itemize}
\end{inprogress}

\subsection{Protocol-Level Network Policy}

\begin{inprogress}
  \begin{itemize}
    \item
  \end{itemize}
\end{inprogress}

\subsection{\bpfcontain{} Policy Generation and \glsentryshort{gui}}

\begin{inprogress}
  \begin{itemize}
    \item
  \end{itemize}
\end{inprogress}



\section{Conclusion}
\label{s:disc-conclusion}

\todo{This section will conclude the thesis, highlighting the important aspects of
\bpfbox{} and \bpfcontain{} and contributions}


% \chapter{Conclusion}%
% \label{c:conclusion}
% \input{chapters/conclusion}

\cleardoublepage%
\printbibliography[heading=bibintoc]%
%\nocite{*} % TODO: Remove this when finished

\appendix%
\printglossary[type=\acronymtype, title=List of Acronyms, toctitle=List of Acronyms]

\end{document}
